\documentclass[a4paper,ngerman,12pt]{scrartcl}

\usepackage[utf8]{inputenc}
%\usepackage[ansinew]{inputenc}

\usepackage[ngerman]{babel}

\usepackage{amsmath,amsthm,amssymb,mathtools,stmaryrd,color,graphicx}
\usepackage{setspace}
\usepackage{bussproofs}
\usepackage{array}
\usepackage{comment}
\usepackage{wrapfig}

\usepackage{enumitem}

\usepackage{units}

\usepackage[protrusion=true,expansion=true]{microtype}

\usepackage{lmodern}

\usepackage{hyperref}
\usepackage{cleveref}

\newcommand{\IR}{\mathbb{R}}
\newcommand{\IC}{\mathbb{C}}
\newcommand{\IZ}{\mathbb{Z}}
\newcommand{\IN}{\mathbb{N}}
\newcommand{\IQ}{\mathbb{Q}}

\setlength\parskip{\medskipamount}
\setlength\parindent{0pt}

\theoremstyle{definition}
\newtheorem{defn}{Definition}[]
\newtheorem{axiom}[defn]{Axiom}
\newtheorem{bsp}[defn]{Beispiel}

\RequirePackage{framed}
\newtheorem{aufg}{Aufgabe}
\definecolor{shadecolor}{rgb}{.96,.96,.96}
\newenvironment{aufgabe}[1][]
		{\begin{shaded}\vspace{-0.3cm}\begin{aufg}\emph{#1} \par\medskip}
		{\end{aufg}\vspace{-0.3cm}\end{shaded}}
\newtheorem{zaufg}{Zusatzaufgabe}
	
\newenvironment{spiel}[1][]{\begin{framed}\textbf{#1:}\\}{\end{framed}}


\theoremstyle{plain}
\newtheorem{prop}[defn]{Proposition}
\newtheorem{motto}[defn]{Motto}
\newtheorem{wunder}[defn]{Wunder}
\newtheorem{ueberlegung}[defn]{Überlegung}
\newtheorem{lemma}[defn]{Lemma}
\newtheorem{kor}[defn]{Korollar}
\newtheorem{hilfsaussage}[defn]{Hilfsaussage}
\newtheorem{satz}[defn]{Satz}
\newtheorem{frage}[defn]{Frage}

\theoremstyle{remark}
\newtheorem{bem}[defn]{Bemerkung}
\newtheorem{beob}[defn]{Beobachtung}

	
\newtheorem*{antwort}{Antwort}

%\newlength{\aufgabenskip}
%\setlength{\aufgabenskip}{1.4em}
%\newcounter{aufgabennummer}
%\newenvironment{aufgabe}[1]{
%	\addtocounter{aufgabennummer}{1}
%	\textbf{Aufgabe \theaufgabennummer.} \emph{#1} \par
%}{\vspace{\aufgabenskip}}

\clubpenalty=10000
\widowpenalty=10000
\displaywidowpenalty=10000

\setlength\unitlength{1cm}

\usepackage{tikz}
\usetikzlibrary{calc}
\usepackage{tkz-euclide}
\usepackage{adjustbox}
\usepackage{algorithm2e}
\usepackage{pgfplots}

\RequirePackage{geometry}
\geometry{textwidth=17.0cm,textheight=25cm,footskip=1.5cm}


\newcommand{\kante}[2]{#1{-}#2}
\newcommand{\edge}[3]{\draw[thick] (#1) --node[rectangle,fill=gray!10]{$#3$} (#2);}

\begin{document}\pagestyle{empty}
	
	\begin{center}
		\begin{tikzpicture}
			\draw[thick] (0,0) -- ++(1,0) -- ++(0,1) -- ++(-.5,.5) -- ++(-.5,-.5) -- cycle;
			\draw[thick] (5.5,0) -- ++(1,0) -- ++(0,1) -- ++(-.5,.5) -- ++(-.5,-.5) -- cycle;
			
			\draw[thick] (-3,-6) --node[above=.2]{$W$} ++(2,0) -- ++(0,1) -- ++(-.3,.3) -- ++(0,-.3) -- ++(-.3,.3) -- ++(0,-.3) -- ++(-.3,.3) -- ++(0,-.3) -- ++(-.3,.3) -- ++(0,-.3) -- ++(-.2,0) -- ++(-.1,1) -- ++(-.3,0) -- cycle;
			
			\draw[thick] (2.5,-6) --node[above=.2]{$G$} ++(2,0) -- ++(0,1) -- ++(-.3,.3) -- ++(0,-.3) -- ++(-.3,.3) -- ++(0,-.3) -- ++(-.3,.3) -- ++(0,-.3) -- ++(-.3,.3) -- ++(0,-.3) -- ++(-.2,0) -- ++(-.1,1) -- ++(-.3,0) -- cycle;
			
			\draw[thick] (8,-6) --node[above=.2]{$S$} ++(2,0) -- ++(0,1) -- ++(-.3,.3) -- ++(0,-.3) -- ++(-.3,.3) -- ++(0,-.3) -- ++(-.3,.3) -- ++(0,-.3) -- ++(-.3,.3) -- ++(0,-.3) -- ++(-.2,0) -- ++(-.1,1) -- ++(-.3,0) -- cycle;
		\end{tikzpicture}
	\end{center}	

	\vspace{5cm}

	\begin{center}
		\begin{tikzpicture}
			\draw[thick] (-2.5,0) -- ++(1,0) -- ++(0,1) -- ++(-.5,.5) -- ++(-.5,-.5) -- cycle;
			\draw[thick] (3,0) -- ++(1,0) -- ++(0,1) -- ++(-.5,.5) -- ++(-.5,-.5) -- cycle;
			\draw[thick] (8.5,0) -- ++(1,0) -- ++(0,1) -- ++(-.5,.5) -- ++(-.5,-.5) -- cycle;
			
			\draw[thick] (-3,-6) --node[above=.2]{$W$} ++(2,0) -- ++(0,1) -- ++(-.3,.3) -- ++(0,-.3) -- ++(-.3,.3) -- ++(0,-.3) -- ++(-.3,.3) -- ++(0,-.3) -- ++(-.3,.3) -- ++(0,-.3) -- ++(-.2,0) -- ++(-.1,1) -- ++(-.3,0) -- cycle;
			
			\draw[thick] (2.5,-6) --node[above=.2]{$G$} ++(2,0) -- ++(0,1) -- ++(-.3,.3) -- ++(0,-.3) -- ++(-.3,.3) -- ++(0,-.3) -- ++(-.3,.3) -- ++(0,-.3) -- ++(-.3,.3) -- ++(0,-.3) -- ++(-.2,0) -- ++(-.1,1) -- ++(-.3,0) -- cycle;
			
			\draw[thick] (8,-6) --node[above=.2]{$S$} ++(2,0) -- ++(0,1) -- ++(-.3,.3) -- ++(0,-.3) -- ++(-.3,.3) -- ++(0,-.3) -- ++(-.3,.3) -- ++(0,-.3) -- ++(-.3,.3) -- ++(0,-.3) -- ++(-.2,0) -- ++(-.1,1) -- ++(-.3,0) -- cycle;
		\end{tikzpicture}
	\end{center}	

	\begin{center}
		\begin{tikzpicture}
			\node[draw,circle] (1)at(0,0) {$1$};
			\node[draw,circle] (2)at(2,0) {$2$};
			\node[draw,circle] (3)at(0,-2) {$3$};
			\node[draw,circle] (4)at(2,-2) {$4$};
			
			\draw[ultra thick] (1) -- (2);
			\draw[ultra thick] (1) -- (3);
			\draw[ultra thick] (1) -- (4);
			\draw[ultra thick] (2) -- (3);
			\draw[ultra thick] (2) -- (4);
			\draw[ultra thick] (3) -- (4);
		\end{tikzpicture}
		\hspace{3em}
		\begin{tikzpicture}
			\node[draw,circle] (1)at(0,0) {$1$};
			\node[draw,circle] (2)at(2,0) {$2$};
			\node[draw,circle] (3)at(-1,-2) {$3$};
			\node[draw,circle] (4)at(3,-2) {$4$};
			\node[draw,circle] (5)at(0,-4) {$5$};
			\node[draw,circle] (6)at(2,-4) {$6$};
			
			\draw[ultra thick] (1) -- (2);
			\draw[ultra thick] (1) -- (3);
			\draw[ultra thick] (1) -- (4);
			\draw[ultra thick] (2) -- (3);
			\draw[ultra thick] (2) -- (4);
			\draw[ultra thick] (2) -- (5);
			\draw[ultra thick] (2) -- (6);
			\draw[ultra thick] (3) -- (4);
			\draw[ultra thick] (3) -- (5);
			\draw[ultra thick] (3) -- (6);
			\draw[ultra thick] (4) -- (5);
			\draw[ultra thick] (5) -- (6);
		\end{tikzpicture}
		\hspace{3em}
		\begin{tikzpicture}
			\node[draw,circle] (1)at(0,0) {$1$};
			\node[draw,circle] (2)at(2,0) {$2$};
			\node[draw,circle] (3)at(-1,-2) {$3$};
			\node[draw,circle] (4)at(3,-2) {$4$};
			\node[draw,circle] (5)at(0,-4) {$5$};
			\node[draw,circle] (6)at(2,-4) {$6$};
			
			\draw[ultra thick] (1) -- (2);
			\draw[ultra thick] (1) -- (3);
			\draw[ultra thick] (1) -- (6);
			\draw[ultra thick] (2) -- (4);
			\draw[ultra thick] (2) -- (5);
			\draw[ultra thick] (3) -- (4);
			\draw[ultra thick] (3) -- (5);
			\draw[ultra thick] (4) -- (6);
			\draw[ultra thick] (5) -- (6);
		\end{tikzpicture}
		
		\vspace{2em}
		\begin{tikzpicture}
			\node[draw,circle] (1)at(0,0) {$1$};
			\node[draw,circle] (2)at(2,1) {$2$};
			\node[draw,circle] (3)at(2,-1) {$3$};
			\node[draw,circle] (4)at(4,-1) {$4$};
			\node[draw,circle] (5)at(4,1) {$5$};
			\node[draw,circle] (6)at(5,-2) {$6$};
			\node[draw,circle] (7)at(7,-2) {$7$};
			\node[draw,circle] (8)at(7,-.5) {$8$};
			\node[draw,circle] (9)at(6,1) {$9$};
			
			\draw[ultra thick] (1) -- (2);
			\draw[ultra thick] (1) -- (5);
			\draw[ultra thick] (2) -- (3);
			\draw[ultra thick] (3) -- (5);
			\draw[ultra thick] (3) -- (6);
			\draw[ultra thick] (4) -- (5);
			\draw[ultra thick] (4) -- (7);
			\draw[ultra thick] (5) -- (8);
			\draw[ultra thick] (5) -- (9);
			\draw[ultra thick] (6) -- (9);
			\draw[ultra thick] (8) -- (9);
		\end{tikzpicture}
		\hspace{3em}
		\begin{tikzpicture}
			\node[draw,circle] (1)at(0,0) {$1$};
			\node[draw,circle] (2)at(2,0) {$2$};
			\node[draw,circle] (3)at(4,0) {$3$};
			\node[draw,circle] (4)at(6,0) {$4$};
			\node[draw,circle] (5)at(-1,-1) {$5$};
			\node[draw,circle] (6)at(0,-2) {$6$};
			\node[draw,circle] (7)at(2,-2) {$7$};
			\node[draw,circle] (8)at(4,-2) {$8$};
			
			\draw[ultra thick] (1) -- (5);
			\draw[ultra thick] (1) -- (6);
			\draw[ultra thick] (1) -- (7);
			\draw[ultra thick] (1) -- (8);
			\draw[ultra thick] (2) -- (6);
			\draw[ultra thick] (2) -- (7);
			\draw[ultra thick] (2) -- (8);
			\draw[ultra thick] (2) -- (3);
			\draw[ultra thick] (3) -- (4);
			\draw[ultra thick] (3) -- (8);
			\draw[ultra thick] (4) -- (6);
			\draw[ultra thick] (4) -- (7);
			\draw[ultra thick] (4) -- (8);
			\draw[ultra thick] (5) -- (6);
		\end{tikzpicture}
	\end{center}	

	\vspace{1cm}

	\begin{center}\renewcommand{\arraystretch}{3}
		\begin{tabular}{c||c|c|c||c}
			\hspace{3em}Graph\hspace{3em} & \hspace{2em}Ecken\hspace{2em} & \hspace{2em}Kanten\hspace{2em} & \hspace{2em}Flächen\hspace{2em} & \phantom{Ergebnis}\\\hline
			& & & & \\
			& & & & \\
			& & & & \\
			& & & & \\
			& & & & \\
			& & & & \\
			& & & & \\
			& & & & \\
		\end{tabular}
	\end{center}

\end{document}