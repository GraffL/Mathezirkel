\documentclass[a4paper,ngerman,12pt]{scrartcl}

\usepackage[utf8]{inputenc}
%\usepackage[ansinew]{inputenc}

\usepackage[ngerman]{babel}

\usepackage{amsmath,amsthm,amssymb,stmaryrd,color,graphicx}
\usepackage{setspace}
\usepackage{bussproofs}
\usepackage{array}
\usepackage{comment}
\usepackage{wrapfig}

\usepackage{enumitem}

\usepackage{units}

\usepackage[protrusion=true,expansion=true]{microtype}

\usepackage{lmodern}

\usepackage{hyperref}
\usepackage{cleveref}

\newcommand{\RR}{\mathbb{R}}
\newcommand{\CC}{\mathbb{C}}
\newcommand{\ZZ}{\mathbb{Z}}
\newcommand{\NN}{\mathbb{N}}
\newcommand{\QQ}{\mathbb{Q}}

\setlength\parskip{\medskipamount}
\setlength\parindent{0pt}

\theoremstyle{definition}
\newtheorem{defn}{Definition}[]
\newtheorem{axiom}[defn]{Axiom}
\newtheorem{bsp}[defn]{Beispiel}

\theoremstyle{plain}
\newtheorem{prop}[defn]{Proposition}
\newtheorem{motto}[defn]{Motto}
\newtheorem{wunder}[defn]{Wunder}
\newtheorem{ueberlegung}[defn]{Überlegung}
\newtheorem{lemma}[defn]{Lemma}
\newtheorem{kor}[defn]{Korollar}
\newtheorem{hilfsaussage}[defn]{Hilfsaussage}
\newtheorem{satz}[defn]{Satz}
\newtheorem{frage}[defn]{Frage}

\theoremstyle{remark}
\newtheorem{bem}[defn]{Bemerkung}
\newtheorem{aufg}[defn]{Aufgabe}

\newtheorem*{antwort}{Antwort}

\newlength{\aufgabenskip}
\setlength{\aufgabenskip}{1.4em}
\newcounter{aufgabennummer}
\newenvironment{aufgabe}[1]{
	\addtocounter{aufgabennummer}{1}
	\textbf{Aufgabe \theaufgabennummer.} \emph{#1} \par
}{\vspace{\aufgabenskip}}

\clubpenalty=10000
\widowpenalty=10000
\displaywidowpenalty=10000

\setlength\unitlength{1cm}

\usepackage{tikz}

\RequirePackage{geometry}
\geometry{textwidth=16.0cm,textheight=24.5cm,footskip=1.5cm}

\usepackage{todonotes}

\begin{document}
	
\begin{picture}(0,0)
\put(0,-0.5){%
	\includegraphics[scale=0.1]{logo-ifm}
}
\put(14.0,-3.5){%
	\includegraphics[scale=0.17]{cover}
}
\end{picture} 
	
\vspace{6em}

\section*{Erste Beweise mit Induktion - Lösungshinweise}

Dieses Skript enthält Lösungs\emph{hinweise} zum letzten Korrespondenzbrief. Manchmal sind dies schon die kompletten Lösungen der Aufgaben, meistens sind es aber nur einige Hinweise, die dir dabei helfen sollen, auch die Aufgaben lösen zu können, bei denen du bisher nicht weiter gekommen bist. Wenn du noch weitere Fragen zu den Aufgaben hast, kannst du uns diese weiterhin gerne per E-Mail stellen.

Wenn du uns bereits deine eigenen Lösungsversuche geschickt hast (oder noch schicken wirst - das ist selbstverständlich immer noch möglich), dann versuchen wir natürlich auch dir mit unseren Korrekturen beim Verständnis der Aufgaben zu helfen. Es lohnt sich also uns deine Lösungen zu senden :-)

\begin{aufgabe}{Der nächste Schritt}
	\missingfigure{Bild des fünften Schrittes}
\end{aufgabe}

\begin{aufgabe}{Wie viele neue Münzen?}
	$4\cdot (n+1)$ viele Münzen.
	
	Im obigen Beweis wird an der Stelle $(\ast)$ nur gesagt, dass die Zahl der neuen Münzen durch $4$ teilbar ist. Kannst du sogar herausfinden, wie viele Münzen es genau sind (in Abhängigkeit von $n$)?
	
	Anders gesagt: Wie viele zusätzliche Münzen benötigt man, um aus einen Viereck mit Seitenlänge $n$ eines mit Seitenlänge $n+1$ zu machen?
\end{aufgabe}

\begin{aufgabe}{Zentrierte Dreieckszahl}
	...
	
	Zeige, dass die zentrierte Dreieckszahl nie durch $3$ teilbar ist (du kannst sogar noch genauer zeigen, dass sie immer um $1$ größer ist als eine durch $3$ teilbare Zahl).
\end{aufgabe}


\end{document}