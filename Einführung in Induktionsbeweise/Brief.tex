\documentclass[a4paper,ngerman,12pt]{scrartcl}

\usepackage[utf8]{inputenc}
%\usepackage[ansinew]{inputenc}

\usepackage[ngerman]{babel}

\usepackage{amsmath,amsthm,amssymb,stmaryrd,color,graphicx}
\usepackage{setspace}
\usepackage{bussproofs}
\usepackage{array}
\usepackage{comment}
\usepackage{wrapfig}

\usepackage{enumitem}

\usepackage{units}

\usepackage[protrusion=true,expansion=true]{microtype}

\usepackage{lmodern}

\usepackage{hyperref}
\usepackage{cleveref}

\newcommand{\RR}{\mathbb{R}}
\newcommand{\CC}{\mathbb{C}}
\newcommand{\ZZ}{\mathbb{Z}}
\newcommand{\NN}{\mathbb{N}}
\newcommand{\QQ}{\mathbb{Q}}

\setlength\parskip{\medskipamount}
\setlength\parindent{0pt}

\theoremstyle{definition}
\newtheorem{defn}{Definition}[]
\newtheorem{axiom}[defn]{Axiom}
\newtheorem{bsp}[defn]{Beispiel}

\theoremstyle{plain}
\newtheorem{prop}[defn]{Proposition}
\newtheorem{motto}[defn]{Motto}
\newtheorem{wunder}[defn]{Wunder}
\newtheorem{ueberlegung}[defn]{Überlegung}
\newtheorem{lemma}[defn]{Lemma}
\newtheorem{kor}[defn]{Korollar}
\newtheorem{hilfsaussage}[defn]{Hilfsaussage}
\newtheorem{satz}[defn]{Satz}
\newtheorem{frage}[defn]{Frage}

\theoremstyle{remark}
\newtheorem{bem}[defn]{Bemerkung}
\newtheorem{aufg}[defn]{Aufgabe}

\newtheorem*{antwort}{Antwort}

\newlength{\aufgabenskip}
\setlength{\aufgabenskip}{1.4em}
\newcounter{aufgabennummer}
\newenvironment{aufgabe}[1]{
	\addtocounter{aufgabennummer}{1}
	\textbf{Aufgabe \theaufgabennummer.} \emph{#1} \par
}{\vspace{\aufgabenskip}}

\clubpenalty=10000
\widowpenalty=10000
\displaywidowpenalty=10000

\setlength\unitlength{1cm}

\usepackage{tikz}

\RequirePackage{geometry}
\geometry{textwidth=16.0cm,textheight=24.5cm,footskip=1.5cm}

\usepackage{todonotes}


\begin{document}
	
\begin{picture}(0,0)
\put(0,-0.5){%
	\includegraphics[scale=0.1]{logo-ifm}
}
\put(14.0,-3.5){%
	\includegraphics[scale=0.17]{cover}
}
\end{picture} 
	
\vspace{6em}

\begin{center}\Large{Vierter Korrespondenzbrief}\end{center}

\section*{Erste Beweise mit Induktion}

In diesem Brief werden wir eine besondere, in der Mathematik sehr oft genutzte Beweistechnik kennen lernen: Den \emph{Beweis durch Induktion}. Eine Besonderheit dieser Beweistechnik ist, dass man mit ihr nicht nur eine Aussage, sondern gleich unendlich viele Aussagen auf einmal zeigen kann! 

\subsection{Induktion mit Bildern}

Vor kurzem habe ich mal wieder mei  Sparschein geleert um es zur Bank zu bringen. Davor wollte ich aber natürlich wissen, wie viel Geld sich überhaupt darin befunden hatte. Und um mir das Zählen zu erleichtern (und vielleicht auch, weil mir gerade ein wenig langweilig war) habe ich die Münzen zu einem Muster zusammengelegt:

\missingfigure{Die ersten vier(?) zentrierten Viereckszahlen}

\begin{aufgabe}{Der nächste Schritt}
	Erkennst du das Schema, nach dem ich dieses Muster gebaut habe? Kannst du das nächste Viereck zeichnen?
\end{aufgabe}

Während ich also immer größere Vierecke gelegt habe, ist mir das folgende aufgefallen: Die Zahl an Münzen, die ich für ein Viereck benötige, scheint immer eine ungerade Zahl zu sein: Für das erste $1$, beim zweiten $5$, beim dritten $13$, beim vierten $25$, beim fünften \underline{\phantom{ 41 }}, \dots

Ob das wohl für alle solchen Vierecke stimmt? Ich habe noch ein paar weitere solchen Vierecke gelegt, und für tatsächlich habe ich für jedes eine ungerade Anzahl an Münzen gebraucht. Aber irgendwann sind mir natürlich die Münzen ausgegangen. Wie kann ich nun also herausfinden, ob meine Vermutung wirklich für \emph{alle} solche Vierecke gilt?

Wir haben nun also für ein paar kleine Beispiele gesehen, dass die Aussage \glqq Für ein (zentriertes) Viereck benötigt man eine ungerade Zahl an Münzen.\grqq{} für diese stimmt. Wir können nun aber nicht \emph{alle} möglichen Vierecke durchprobieren und für jedes einzelne prüfen, ob die Aussage auch für diese stimmt.

Ein ähnliches Problem hatten wir zu Beginn dieses Schuljahres schon einmal in einem Zirkelbrief? Erinnerst du dich noch an den Brief mit den Unmöglichkeitsbeweisen? Den Tetrominos, den Schachbrettern und den Chamäleons? 

\missingfigure{Kleines Bild von Tetrominos, Schachbrettern und Chamäleons?}

Darin hatten wir doch ein ganz ähnliches Problem: Wir hatten zum Beispiel eine Menge verschiedenfarbiger Chamäleons, die ihre Farbe nach einem bestimmten Muster ändern, wenn sie sich treffen. Wir wollten nun zeigen, dass wir nie gleich viele Chamäleons jeder Farbe bekommen können.

\todo[inline]{Wie ging es weiter (Invariante!)}

Ganz genauso wollen wir nun auch mit unseren Münzvierecken vorgehen, um zu zeigen, dass wir immer eine ungerade Anzahl von Münzen benötigen:

\begin{description}
	\item[Anfang:] Für das kleinstmögliche Viereck (mit Seitenlänge $1$) ist die Aussage klar: Es besteht aus genau einer Münze - und $1$ ist eine ungerade Zahl.
	\item[Voraussetzung:] Wir haben ein Viereck mit Seitenlänge $n$, für das wir bereits wissen, dass wir für es ungerade viele Münzen benötigen.
	\item[Schlussfolgerung:] Wir wollen nun zeigen, dass auch ein Viereck mit Seitenlänge $n+1$ ungerade viele Münzen benötigt. Dazu nehmen wir unser Viereck mit Seitenlänge $n$ und machen daraus ein Viereck der Seitenlänge $n+1$:
	\missingfigure{Viereck $n$ zu Viereck $n+1$}
	Was müssen wir dazu also machen? Wir fügen auf jeder Seite zusätzlich eine Reihe Münzen hinzu $(\ast)$. Da wir nun auf jeder der vier Seiten gleich viele Münzen hinzufügen, ist die Zahl der neuen Münzen sicher durch $4$ teilbar, also eine gerade Zahl.
	
	Die Gesamtzahl der Münzen im $(n+1)$-Viereck ist also gleich der Zahl der Münzen im $n$-ten Viereck (ungerade!) plus der Zahl der neuen Münzen (gerade!). Da nun die Summe einer geraden und einer ungeraden Zahl wieder eine ungerade bildet
	\footnote{Du bist skeptisch, ob diese Aussage wirklich für alle Zahlen gilt? Sehr gut! In der Mathematik sollte man nie irgendwelche Behauptungen einfach so glauben, sondern sie am besten selbst nachprüfen. Prüfe die Aussage am besten mal an ein paar Beispielen nach. Wenn du sie wirklich beweisen willst, dann schau dir noch einmal den zweiten Korrespondenzbrief (zu Teilbarkeit) an - darin findest du alle dafür notwendigen Werkzeuge.},
	ist also auch die Zahl der Münzen in unserem größeren Viereck erneut eine ungerade Zahl.
\end{description}

Damit haben wir nun bewiesen, dass alle nach dem beschriebenen Schema gelegten Vierecke eine ungerade Anzahl von Münzen benötigen. Warum?
\todo[inline]{Erklärung: Viereck kann gebaut werden, indem wir mit einem kleinen anfagen (ungerade Zahl Münzen nach Anfang), dann das Viereck um einen Schritt größer machen (weiterhin ungerade Zahl nach Schlussfolgerung). Damit gilt die Voraussetzung und wir können beim erneuten Größermachen wieder die Schlussfolgerung anwenden. Immer so weiter bis wir das gewünschte Viereck erreicht haben. Während dem ganzen schrittweisen Bauen blieb die Aussage ungerade Münzen immer wahr. Also ist sie es auch am Ende 
noch}

\begin{aufgabe}{Wie viele neue Münzen?}
	Im obigen Beweis wird an der Stelle $(\ast)$ nur gesagt, dass die Zahl der neuen Münzen durch $4$ teilbar ist. Kannst du sogar herausfinden, wie viele Münzen es genau sind (in Abhängigkeit von $n$)?
	
	Anders gesagt: Wie viele zusätzliche Münzen benötigt man, um aus einen Viereck mit Seitenlänge $n$ eines mit Seitenlänge $n+1$ zu machen?
\end{aufgabe}

\begin{aufgabe}{Zentrierte Dreieckszahl}
	\missingfigure{zentrierte Dreieckszahlen}
	
	Zeige, dass die zentrierte Dreieckszahl nie durch $3$ teilbar ist (du kannst sogar noch genauer zeigen, dass sie immer um $1$ größer ist als eine durch $3$ teilbare Zahl).
\end{aufgabe}

\subsection{Induktion mit Zahlen}

\todo[inline]{Begriff Induktionsbeweis einführen, etwas formalere Beschreibung}

\todo[inline]{Formaler Induktionsbeweis für: Beweise für diese Formel: Zahl ist für alle $n$ ungerade}

\begin{aufgabe}{Durch $3$ teilbar?}
	Dreieckszahl minus $1$
\end{aufgabe}

\begin{aufgabe}{Eine alte Bekannte}
	Weiteres dazu: Beweise n-te Viereckszahl ist $2n^2-2n+1$.
\end{aufgabe}

\begin{aufgabe}{Titel}
	Beweise: Jede Viereckszahl ist Summe zweier aufeinander folgender Quadratzahlen (evtl. auch durch Bild?)
\end{aufgabe}

\hrule

Formalisierung: Induktionsanfang + Induktionsvoraussetzung + Induktionsschritt

\hrule

Beweise: Die Länge der Kochschen Schneeflocke ist unendlich bzw. besser: Nach $n$ Schritten hat sie Länge $> n$

Analog: Die Fläche des Sierpinski-Dreiecks ist $0$

\hrule

Josephus-Problem

\hrule

Beweise über Bäume? Evtl. als "echte" Bäume beschreiben und nur kurzer Hinweis auf Graphentheorie?

\hrule

Induktion verwendet man auch für Korrektheitsbeweise von Computerprogrammen:

Beispiel: Quadratzahl $x^2$ berechnen als $x+x-1 +x-1+x-2 + \dots +1$ - evtl. dargestellt als Flussdiagramm (rekursives Programm!)

Aufgabe?

Euklidischer Algorithmus? Verweis auf zweiten Brief

\hrule

\begin{bem}
Lauter Aussagen über natürliche Zahlen. Dies liegt daran, dass die natürlichen Zahlen selbst mit Hilfe von Induktion definiert sind. Verweis auf Peano-Axiome
\end{bem}

\end{document}