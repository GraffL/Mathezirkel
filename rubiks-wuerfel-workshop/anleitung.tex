\documentclass[12pt]{scrartcl}

\usepackage[utf8]{inputenc}
\usepackage[ngerman]{babel}

\usepackage{xcolor} % Farben
\usepackage{mdframed} % Rahmen für die Algorithmen

\usepackage{amsthm} % Theorem-Umgebungen
\newcounter{fallCounter}
\theoremstyle{definition}
\newtheorem{fall}[fallCounter]{Fall}
\newcounter{trickCounter}
\newtheorem{trickk}[trickCounter]{Trick}
% Jeder Trick ist Trick 17 :-)
\newenvironment{trick}
  {\setcounter{trickCounter}{16}\begin{trickk}}
  {\end{trickk}}


% Für die Verwendung von `diagrams` mit LaTeX
\usepackage[backend=cairo, outputdir=diagrams]{diagrams-latex}
\usepackage{graphicx}

% Weniger Abstand nach oben und unten
\usepackage[hscale=0.75,vscale=0.92,vmarginratio={1:1},heightrounded]{geometry}

% Seitennummerierung ausschalten
\thispagestyle{empty}
\pagestyle{empty}

% Überschriften-Größen
\setkomafont{section}{\Huge}
\setkomafont{subsection}{\LARGE}
\setkomafont{subsubsection}{\Large}

\newenvironment{algorithm}
  {\setcounter{fallCounter}{0}\vspace{15pt}\begin{mdframed}[backgroundcolor=blue!15]}
  {\end{mdframed}\vspace{15pt}}

\setcounter{secnumdepth}{0} % Keine Abschnittsnummerierung

\begin{document}

\section{Station 1 -- Die obere Ebene}

\subsection{Vorbemerkungen}

Zunächst eine wichtige Beobachtung: Egal, welche Drehungen man ausführt, die mittleren Würfelchen auf allen Seiten ändern ihre Position zueinander nicht. Wir werden diese mittleren Würfel als fest betrachten und mit der "`grünen Seite"' (oder blauen, gelben) immer diejenige Seite bezeichnen, deren mittleres Würchelchen grün ist.

Wir lösen den Würfel nun Ebene für Ebene. Bevor du anfängst, solltest du dir eine Seite aussuchen, mit der du anfangen möchtest.
In dieser Anleitung haben wir die grüne Seite gewählt (also die mit dem grünen mittleren Würfelchen). Wenn du nicht unnötig umdenken möchtest, dann solltest du auch die grüne Seite nehmen. Halte den Würfel ab jetzt so, dass sie oben ist.

Noch etwas: Während du einen der angegebenen Algorithmen ausführst, darfst du den Würfel insgesamt nicht drehen!
Wenn am Anfang der Ausführung die grüne Seite oben ist und die orange Seite zu dir schaut, dann muss das während der gesamten Ausführung des Algorithmus auch so bleiben!

Wenn du gerade keinen Algorithmus ausführst, darfst du den Würfel um die "`Nord-Süd-Achse"' drehen.

\pagebreak

\subsection{1.A \enspace Die Kanten der oberen Ebene}

Unser erstes Ziel ist es, die Kanten der obersten Ebene richtig zu stellen. Der Würfel soll danach also folgendermaßen aussehen:

\begin{center}
  \begin{diagram}[width=300,height=100]
    import Diagrams.RubiksCube
    import Control.Lens hiding ((#))
    c = (pure (const lightgray) & topLayerFacets .~ id & centerFacets .~ id & cornerFacets .~ (const lightgray)) <*> solvedRubiksCube
    dia = [ scale 0.7 (drawFoldingPattern (c ^. rotateDown))
          , drawRubiksCube with c
          ] # map center # hcat' (with & sep .~ 1.5) # pad 1.05
  \end{diagram}
\end{center}

(Graue Flächen stehen hier für beliebige, unsortierte Würfelchen.)

Dies können wir folgendermaßen bewerkstelligen: Wir suchen zunächst eine Kante, die momentan nicht an ihrem richtigen Platz in der oberen Ebene ist. Durch geschickte Drehungen kannst du erreichen, dass diese Kante in der untersten Ebene ist. Wenn das der Fall ist, dann kannst du die untere Ebene so lange drehen, bis die Kante unter dem Platz ist, wo sie eigentlich hin muss, sodass man sie nur noch "`hochheben"' muss. Genau das erledigt dann der folgende Algorithmus:

\begin{algorithm}
  \subsubsection{Der Kanten-Hebungs-Algorithmus}
  \begin{fall}
    Die grüne Fläche zeigt nach unten (d.\,h. die Kante ist richtig orientiert). Dann muss man nur die vordere Seite um 180 Grad drehen:
    \begin{center}
      \begin{diagram}[width=300,height=70]
        import Diagrams.RubiksCube
        c = (pure (const lightgray) & topLayerFacets .~ id & centerFacets .~ id & cornerFacets .~ (const lightgray)) <*> solvedRubiksCube
        moves = [F, F]
        dia = drawMovesBackward (with & showStart .~ True) c moves # pad 1.05
      \end{diagram}
    \end{center}
  \end{fall}
%\end{algorithm}
%\begin{algorithm}
  \begin{fall}
    Die grüne Fläche zeigt nach vorne. Dann haben wir etwas mehr zu tun:
    \iffalse
    \begin{center}
      \begin{diagram}[width=450,height=70]
        import Diagrams.RubiksCube
        c = (pure (const lightgray) & topLayerFacets .~ id & centerFacets .~ id & cornerFacets .~ (const lightgray)) <*> solvedRubiksCube
        moves = [R, F', R', D', F, F]
        dia = drawMovesBackward (with & showStart .~ True) c moves # pad 1.05
      \end{diagram}
    \end{center}
    Mit den ersten vier Drehungen schaffen wir es dabei, den Würfel richtig zu orientieren und können Fall 1 anwenden.
    Wir haben ein neues Problem auf ein bekanntes zurückgeführt.
    \fi

    \begin{center}
      \begin{diagram}[width=450,height=70]
        import Diagrams.RubiksCube
        c = (pure (const lightgray) & topLayerFacets .~ id & centerFacets .~ id & cornerFacets .~ (const lightgray)) <*> solvedRubiksCube
        moves = [R', D, R, F']
        dia = drawMovesBackward (with & showStart .~ True) c moves # pad 1.05
      \end{diagram}
    \end{center}
  \end{fall}
\end{algorithm}

Führe diese Schritte für alle vier Kanten der oberen Ebene aus. Passe auf, dass du dabei nicht die schon geleistete Arbeit durch unbedachte Drehungen zerstörst! Es ist oft nötig, \emph{vorübergehend} richtige Kanten von ihrem Platz weg zu drehen. Wenn du das tust, dann solltest du dir immer deine Züge merken, damit du, wenn du nicht mehr weiter weist, deine Züge rückgängig machen kannst und so zumindest nichts zerstört hast.

\pagebreak

\subsection{1.B \enspace Die Ecken der oberen Ebene}

Nun wollen wir auch die Ecken der obersten Ebene richtigstellen. Nach diesem Schritt soll sich folgendes Bild ergeben:

\begin{center}
  \begin{diagram}[width=300,height=80]
    import Diagrams.RubiksCube
    import Control.Lens
    c = (pure (const lightgray) & topLayerFacets .~ id & centerFacets .~ id) <*> solvedRubiksCube
    dia = pad 1.05 $ hcat' (with & sep .~ 1.5) $ map center
            [ scale 0.7 (drawFoldingPattern (c ^. rotateDown))
            , drawRubiksCube with c
            ]
  \end{diagram}
\end{center}

Wir suchen nun eine Ecke mit einer grünen Seitenfläche, die noch nicht in der richtigen Position ist. Wir nehmen zunächst an, dass diese Ecke sich in der unteren Ebene befindet (falls das nicht der Fall ist: lies unter dem Algorithmus weiter). Durch Drehen der unteren Ebene können wir erreichen, dass die Ecke sich genau unter ihrer richtigen Position befindet. Um die Ecke hochzuheben, gibt es

\begin{algorithm}
  \subsubsection{Der Ecken-Hebungs-Algorithmus}
  \begin{fall}
    Die grüne Fläche "`schaut"' zur Seite. Richte dann den Würfel so aus, dass die grüne Fläche vorne ist.

    a) Wenn die grüne Fläche jetzt rechts ist, führe folgende Schritte aus:
    \begin{center}
      \begin{diagram}[width=300,height=50]
        import Diagrams.RubiksCube
        c = (pure (const lightgray) & topLayerFacets .~ id & centerFacets .~ id) <*> solvedRubiksCube
        moves = [D', R', D, R]
        dia = drawMovesBackward (with & showStart .~ True) c moves # pad 1.05
      \end{diagram}
    \end{center}
    b) Wenn die grüne Fläche links ist, musst genau spiegelverkehrt arbeiten:
    \begin{center}
      \begin{diagram}[width=300,height=50]
        import Diagrams.RubiksCube
        c = (pure (const lightgray) & topLayerFacets .~ id & centerFacets .~ id) <*> solvedRubiksCube
        moves = map mirror [D', R', D, R]
        dia = drawMovesBackward (with & showStart .~ True) c moves # pad 1.05
      \end{diagram}
    \end{center}
  \end{fall}

  \begin{fall}
    Die grüne Fläche "`schaut"' nach unten.
    Drehe den gesamten Würfel um die Nord-Süd-Achse, sodass der Eck-Würfel mit der grünen Fläche vorne-unten-rechts ist. Dann:
    \vspace{-10pt}
    \begin{center}
      \begin{diagram}[width=300,height=50]
        import Diagrams.RubiksCube
        import Control.Lens hiding ((#))
        goal = (pure (const lightgray) & topLayerFacets .~ id & centerFacets .~ id) <*> solvedRubiksCube
        movesCase1 = map mirror [D', R', D, R]
        beforeCase1 = goal ^. rotateLeft . undoMoves movesCase1 . rotateRight
        moves = [R', D, R, D', D']
        dia = drawMovesBackward (with & showStart .~ False) beforeCase1 moves # pad 1.05
      \end{diagram}
    \end{center}
    Die grüne Fläche schaut jetzt zur Seite.
    Wenn wir den gesamten Würfel jetzt nach links drehen, können wir also Fall 1b) anwenden.
    Wir haben ein neues Problem auf ein bekanntes zurückgeführt.
  \end{fall}
\end{algorithm}

Verwende diesen Algorithmus, um alle vier Ecken der ersten Ebene zu lösen.

Was aber, wenn die Ecke mit der grünen Seitenfläche schon in der oberen Ebene ist, aber an falscher Stelle oder sogar an richtiger Stelle, aber falsch orientiert? Dann hilft

\begin{trick}
  Halte den Würfel so, dass die Ecke mit der grünen Seitenfläche vorne-oben-rechts ist. Führe dann Fall 1a) des Ecken-Hebungs-Algorithmus aus.
\end{trick}

Dadurch ersetzen wir den Würfel mit der grünen Seitenfläche mit dem Würfel, der sich zufälligerweise gerade vorne-unten-rechts befindet. Der Teil-Würfel mit der grünen Seitenfläche wandert dabei in die untere Ebene. Nach einer Drehung der unteren Ebene können wir den Ecken-Hebungs-Algorithmus auch auf ihn anwenden.

\pagebreak

\section{Station 2 -- Die mittlere Ebene}

Wir wollen die Kanten in der mittleren Ebene korrigieren:

\begin{center}
  \begin{diagram}[width=300,height=100]
    import Diagrams.RubiksCube
    import Control.Lens hiding ((#))
    c = (pure id & bottomLayerFacets .~ (const lightgray) & centerFacets .~ id) <*> solvedRubiksCube
    dia = [ scale 0.7 (drawFoldingPattern (c ^. rotateDown))
          , drawRubiksCube with c
          ] # map center # hcat' (with & sep .~ 1.5) # pad 1.05
  \end{diagram}
\end{center}

Ähnlich zu den letzten beiden Schritten suchen wir in der untersten Ebene nach einer Kante, die eigentlich in die mittlere Ebene gehört.
Hier wäre das zum Beispiel die gelb-orange Kante. Angenommen, die orange Fläche dieser Kante schaut zur Seite.
Wir drehen die untere Ebene so, dass die orange Seitenfläche mit auf der orangen Seite liegt (wie im ersten Bild im Algorithmus). Dann verwenden wir:

\begin{algorithm}
  \subsubsection{Der Kanten-Einsetz-Algorithmus}
  \begin{fall}
    Die Farbe der unteren Fläche (hier gelb) der Kante ist die Farbe der rechten Seite des Würfel (bei Sicht auf die vordere Fläche der Kante).
    Dann:
    \begin{center}
      \begin{diagram}[width=320,height=120]
        import Diagrams.RubiksCube
        import Control.Lens hiding ((#))
        c = (pure id & bottomLayerFacets .~ (const gray) & centerFacets .~ id) <*> solvedRubiksCube
        moves = [D', R', D, R, D, F, D', F']
        moves1 = take 4 moves
        moves2 = drop 4 moves
        afterMoves1 = c ^. undoMoves moves2
        dia = [ drawMovesBackward (with & showStart .~ True & showEnd .~ False) afterMoves1 moves1
              , drawMoves with afterMoves1 moves2
              ] # vcat' (with & sep .~ 0.75) # pad 1.05
      \end{diagram}
    \end{center}
  \end{fall}
  \begin{fall}
    Die Farbe der unteren Fläche (hier weiß) der Kante ist die Farbe der linken Seite des Würfel (bei Sicht auf die vordere Fläche der Kante).
    Dann:
    \begin{center}
      \begin{diagram}[width=320,height=120]
        import Diagrams.RubiksCube
        import Control.Lens hiding ((#))
        c = (pure id & bottomLayerFacets .~ (const gray) & centerFacets .~ id) <*> solvedRubiksCube
        moves = map mirror [D', R', D, R, D, F, D', F']
        moves1 = take 4 moves
        moves2 = drop 4 moves
        afterMoves1 = c ^. undoMoves moves2
        off = Offsets (-0.3) 0.35
        dia = [ drawMovesBackward (with & showStart .~ True & showEnd .~ False & offsets .~ off) afterMoves1 moves1
              , drawMoves (with & offsets .~ off) afterMoves1 moves2
              ] # vcat' (with & sep .~ 0.75) # pad 1.05
      \end{diagram}
    \end{center}
  \end{fall}
\end{algorithm}

Was aber, wenn eine Kante, die in die mittlere Ebene gehört, schon in der mittleren Ebene ist, aber an falscher Stelle oder sogar an richtiger Stelle, aber falsch orientiert? Dann hilft wieder einmal

\begin{trick}
  Halte den Würfel so, dass die falsche Kante vorne-rechts ist. Führe dann Fall 1 des Kanten-Einsetz-Algorithmus aus.
\end{trick}

Die Idee hinter diesem Trick ist die gleiche wie beim letzten Trick.

\pagebreak

\section{Station 3 -- Die letzte Ebene}

Der letzte Schritt ist es, die verbleibende Ebene richtig zu drehen. Damit wir diese Ebene besser im Blick haben, drehen wir den Würfel als erstes um, sodass die \emph{unfertige Ebene oben} ist.

\subsection{3.A \enspace Das Kantenkreuz}

Unser erstes Ziel ist es, auf der noch ungelösten Seite ein Kantenkreuz herzustellen. Das soll in etwa so aussehen:

\begin{center}
  \begin{diagram}[width=320,height=120]
    import Diagrams.RubiksCube
    import Control.Lens hiding ((#))
    blueUp = solvedRubiksCube ^. rotateLeft . rotateLeft . rotateDown . rotateDown
    c = (pure id & cornerFacets .~ (const lightgray) & bottomLayerFacets .~ id) <*> blueUp
    moves = [R, U, U, R', U', R, U', R']
    beforeMoves = c ^. doMoves moves
    dia = [ scale 0.7 (drawFoldingPattern (beforeMoves ^. rotateDown))
          , drawRubiksCube with beforeMoves
          ] # map center # hcat' (with & sep .~ 1.5) # pad 1.05
  \end{diagram}
\end{center}

Das dabei die Kanten noch nicht an den richtigen Positionen sind, ist egal -- worauf es in diesem Schritt ankommt, ist das blaue Kreuz.

\begin{algorithm}
  \subsubsection{Der Kanten-Kipp-Karussell-Algorithmus}
  % XXX: Besserer Name?
  Dieser Algorithmus bewirkt eine zyklische Vertauschung von drei Kantenwürfeln:
  \begin{center}
    \begin{tabular}{ r c l }
      vorne-oben & $\longrightarrow$ & rechts-oben (mit Kippung) \\
      rechts-oben & $\longrightarrow$ & hinten-oben \\
      hinten-oben & $\longrightarrow$ & vorne-oben (mit Kippung)
    \end{tabular}
  \end{center}
  Dabei bedeutet "`mit Kippung"', dass wenn vorher beim Würfel oben-vorne die blaue Seitenfläche an der Seite war, nach Ausführung des Algorithmus diese Seitenfläche oben ist.
  \begin{center}
    \begin{diagram}[width=320,height=110]
      import Diagrams.RubiksCube
      import Control.Lens hiding ((#))
      blueUp = solvedRubiksCube ^. rotateLeft . rotateLeft . rotateDown . rotateDown
      c = (pure id & cornerFacets .~ (const lightgray) & bottomLayerFacets .~ id) <*> blueUp
      moves = [F,R,U,R',U',F']
      moves1 = take 3 moves
      moves2 = drop 3 moves
      afterMoves1 = c ^. undoMoves moves2
      dia = [ drawMovesBackward (with & showStart .~ True & showEnd .~ False) afterMoves1 moves1
            , drawMoves with afterMoves1 moves2
            ] # vcat' (with & sep .~ 0.75) # pad 1.05
    \end{diagram}
  \end{center}
  (Die Farben dienen nur zur Veranschaulichung des Algorithmus. Die Stellung der \\
  Kantenwürfel muss vor Anwendung des Algorithmus \emph{nicht} unbedingt dem Bild \\
  oben links entsprechen.)
\end{algorithm}

Drehe deinen Würfel so, dass er von oben betrachtet wie einer dieser Seiten aussieht:
\begin{center}
  \begin{diagram}[width=320,height=50]
    import Diagrams.RubiksCube
    import Control.Lens hiding ((#))
    s = pure lightgray & middleCenter .~ blue
    dx = r2 (1,0)
    dy = r2 (0,1)
    dia = [ s
          , s & middleLeft .~ blue & middleRight .~ blue
          , s & middleLeft .~ blue & topCenter .~ blue
          ] # map (drawSide dx dy) # hcat' (with & sep .~ 1.5) # pad 1.05
  \end{diagram}
\end{center}
Führe den Algorithmus aus. Wenn du noch nicht fertig bist, wiederhole diese Schritte.

\pagebreak

\subsection{3.B \enspace Die Position der Kanten}

Das Ziel ist nun folgendes Muster:

\begin{center}
  \begin{diagram}[width=320,height=120]
    import Diagrams.RubiksCube
    import Control.Lens hiding ((#))
    blueUp = solvedRubiksCube ^. rotateLeft . rotateLeft . rotateDown . rotateDown
    c = (pure id & cornerFacets .~ (const lightgray) & bottomLayerFacets .~ id) <*> blueUp
    dia = [ scale 0.7 (drawFoldingPattern (c ^. rotateDown))
          , drawRubiksCube with c
          ] # map center # hcat' (with & sep .~ 1.5) # pad 1.05
  \end{diagram}
\end{center}

\begin{algorithm}
  \subsubsection{Der Kanten-Karussell-Algorithmus}
  Dieser Algorithmus bewirkt eine zyklische Vertauschung von drei Kantenwürfeln:
  \begin{center}
    \begin{tabular}{ r c l }
      rechts-oben & $\longrightarrow$ & links-oben \\
      links-oben & $\longrightarrow$ & hinten-oben \\
      hinten-oben & $\longrightarrow$ & rechts-oben
    \end{tabular}
  \end{center}
  \begin{center}
    \begin{diagram}[width=320,height=120]
      import Diagrams.RubiksCube
      import Control.Lens hiding ((#))
      blueUp = solvedRubiksCube ^. rotateLeft . rotateLeft . rotateDown . rotateDown
      c = (pure id & cornerFacets .~ (const lightgray) & bottomLayerFacets .~ id) <*> blueUp
      moves = [R, U, U, R', U', R, U', R']
      moves1 = take 4 moves
      moves2 = drop 4 moves
      afterMoves1 = c ^. undoMoves moves2
      dia = [ drawMovesBackward (with & showStart .~ True & showEnd .~ False) afterMoves1 moves1
            , drawMoves with afterMoves1 moves2
            ] # vcat' (with & sep .~ 0.75) # pad 1.05
    \end{diagram}
  \end{center}
  (Die Farben dienen nur zur Veranschaulichung des Algorithmus. Die Stellung der \\
  Kantenwürfel muss vor Anwendung des Algorithmus \emph{nicht} unbedingt dem Bild \\
  oben links entsprechen.)
\end{algorithm}

Zum Einsatz des Algorithmus:

Versuche, die oberste Ebene so zu drehen, dass \emph{genau eine} Kante an der richtigen Position ist.
Wenn das nicht möglich ist, dann führe den Algorithmus einmal aus (in beliebiger Position) und versuche es erneut.

Halte den Würfel so, dass die richtige Kante oben-vorne ist. Führe den Algorithmus aus. Überprüfe, ob alle Kanten nun richtig sind. Wenn nicht, dann werden sie es durch ein weiteres Ausführen des Algorithmus.

\pagebreak

\subsection{3.C \enspace Die Position der Ecken}

Ziel: Die Eckwürfel der letzten Ebene sind an der richtigen Position, aber möglicherweise nicht richtig orientiert (d.\,h. gedreht). Beispiel:

\begin{center}
  \begin{diagram}[width=320,height=120]
    import Diagrams.RubiksCube
    import Control.Lens hiding ((#))
    simpleMoves = [R, U, U, R', U', R, U', R']
    moves = simpleMoves ++ map mirror simpleMoves
    blueUp = solvedRubiksCube ^. rotateLeft . rotateLeft . rotateDown . rotateDown
    c = blueUp ^. doMoves (moves ++ moves) ^. rotateLeft ^. doMoves moves ^. rotateRight
    dia = [ scale 0.7 (drawFoldingPattern (c ^. rotateDown))
          , drawRubiksCube with c
          ] # map center # hcat' (with & sep .~ 1.5) # pad 1.05
  \end{diagram}
\end{center}

\begin{algorithm}
  \subsubsection{Der Ecken-Karussell-Algorithmus}
  Dieser Algorithmus bewirkt eine zyklische Vertauschung von drei Eckwürfeln:
  \begin{center}
    \begin{tabular}{ r c l }
      vorne-rechts-oben & $\longrightarrow$ & hinten-rechts-oben \\
      hinten-rechts-oben & $\longrightarrow$ & hinten-links-oben \\
      hinten-links-oben & $\longrightarrow$ & vorne-rechts-oben
    \end{tabular}
  \end{center}
  \begin{center}
    \begin{diagram}[width=320,height=100]
      import Diagrams.RubiksCube
      import Control.Lens hiding ((#))
      blueUp = solvedRubiksCube ^. rotateLeft . rotateLeft . rotateDown . rotateDown
      c = blueUp
      moves = [R, U', L', U, R', U', L, U]
      moves1 = take 4 moves
      moves2 = drop 4 moves
      afterMoves1 = c ^. undoMoves moves2
      dia = [ drawMovesBackward (with & showStart .~ True & showEnd .~ False) afterMoves1 moves1
            , drawMoves with afterMoves1 moves2
            ] # vcat' (with & sep .~ 0.75) # pad 1.05
    \end{diagram}
  \end{center}
  (Die Farben dienen nur zur Veranschaulichung des Algorithmus. Die Stellung der \\
  Eckwürfel muss vor Anwendung des Algorithmus \emph{nicht} unbedingt dem Bild \\
  oben links entsprechen.)
\end{algorithm}

Zum Einsatz des Algorithmus:

\setcounter{fallCounter}{0}

\begin{fall}
  Wenn es einen Eckwürfel in der oberen Ebene gibt, der schon an der richtigen Position ist, dann halte den Würfel so, dass dieser Eckwürfel oben links ist. Führe dann den Ecken-Karussell-Algorithmus aus. Überprüfe, ob nun alle Ecken an der richtigen Position sind. Wenn nicht, dann führe den Algorithmus noch einmal aus.
\end{fall}

\begin{fall}
  Wenn es keinen Eckwürfel an der richtigen Position gibt, dann führe den Algorithmus zunächst einmal aus (wie du den Würfel hältst ist egal, Hauptsache blau ist oben). Danach gibt es einen Eckwürfel an der richtigen Position.
\end{fall}

\pagebreak

\subsection{3.D \enspace Die Orientierung der Ecken}

Nur noch dieser Schritt, dann ist es geschafft! Das Kochrezept für diesen Schritt ist das (mehrmalige) Anwenden von

\begin{algorithm}
  \subsubsection{Der Ecken-Kreisel-Algorithmus}
  Dieser Algorithmus bewirkt eine Drehung des Würfelchens oben-vorne-rechts um 120 Grad gegen den UZS
  und gleichzeitig eine Drehung des Würfelchens oben-hinten-rechts um 120 Grad im UZS. Etwa so:

  \begin{center}
    \begin{diagram}[width=300,height=70]
      import Diagrams.RubiksCube
      import Control.Lens hiding ((#))
      blueUp = solvedRubiksCube ^. rotateLeft . rotateLeft . rotateDown . rotateDown
      simpleMoves = [R, U, U, R', U', R, U', R']
      moves = simpleMoves ++ map mirror simpleMoves
      beforeMoves = blueUp ^. undoMoves moves
      dia = [ drawRubiksCube with beforeMoves
            , arrowBetween' (with & headLength .~ veryLarge) (p2 (0,2)) (p2 (3.5,2))
                # dashingG [0.15, 0.15] 0
            , drawRubiksCube with blueUp
            ] # hcat' (with & sep .~ 0.75) # pad 1.05
    \end{diagram}
  \end{center}

  \textbf{Schritte}: \\
  1. Führe den Kanten-Karussell-Algorithmus aus \\
  2. Führe den spiegelverkehrten Kanten-Karussell-Algorithmus aus (siehe unten)
\end{algorithm}

\begin{algorithm}
  \subsubsection{Der spiegelverkehrte Kanten-Karussell-Algorithmus}
  \begin{center}
    \begin{diagram}[width=320,height=120]
      import Diagrams.RubiksCube
      import Control.Lens hiding ((#))
      blueUp = solvedRubiksCube ^. rotateLeft . rotateLeft . rotateDown . rotateDown
      c = (pure id & cornerFacets .~ (const lightgray) & bottomLayerFacets .~ id) <*> blueUp
      moves = map mirror [R, U, U, R', U', R, U', R']
      moves1 = take 4 moves
      moves2 = drop 4 moves
      afterMoves1 = c ^. undoMoves moves2
      dia = [ drawMovesBackward (with & showStart .~ True & showEnd .~ False) afterMoves1 moves1
            , drawMoves with afterMoves1 moves2
            ] # vcat' (with & sep .~ 0.75) # pad 1.05
    \end{diagram}
  \end{center}
\end{algorithm}

\end{document}
