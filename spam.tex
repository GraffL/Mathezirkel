\documentclass{article}

\usepackage[utf8]{inputenc}
\usepackage[ngerman]{babel}
\usepackage[a5paper]{geometry}
\usepackage{parskip} % Keine Einrückung, aber Abstand zwischen Absätzen

\geometry{margin=1cm}

\begin{document}
  \begin{center}
    \huge Spams filtern mit Bayes
  \end{center}
  \vspace{3em}

  \small
  Füllwörter: der, die das, ein, eine, ich, du, Sie, mein, dein, in, für, auf, ist

  \begin{center}\begin{tabular}{r | r | r}
    Wort & Anteil der Ham-Mails, & Anteil der Spam-Mails, \\
         & in denen Wort vorkommt & in denen Wort vorkommt \\[0.5em] \hline
    Geld & 2\% & 4\% \\
    Grüße & 80\% & 80\% \\
    haben & 30\% & 30\% \\
    Konto & 1\% & 5\% \\
    Matthias & 15\% & 0,05\% \\
    Nigeria & 0,1\% & 5\% \\
    überweisen & 1\% & 5\% \\
    Universität & 2\% & 0,1\% \\
    Urlaub & 2\% & 5\% \\
    viele & 50\% & 50\% \\
    wie & 10\% & 10\% \\
  \end{tabular}\end{center}

  \vspace{4em}

  \textbf{Aufgabe}: \enspace Berechne die Spammigkeit folgender Nachrichten: \\[2em]

  \begin{enumerate}
    \item Hast du das Geld für die Universität überwiesen? \\[2em]
    \item Wie ist dein Urlaub in Nigeria? \\\\ Viele Grüße \\\\ Matthias \\[2em]
    \item Überweisen Sie das Geld auf das Konto in Nigeria. \\[2em]
  \end{enumerate}

  Denke dir weitere Sätze aus und berechne deren Spamwahrscheinlichkeit!
\end{document}
