\documentclass[a4paper,ngerman,12pt]{scrartcl}

\usepackage[ngerman]{babel}
\usepackage[utf8]{inputenc}

\usepackage{amsthm,amsmath,amssymb,amsfonts}
\usepackage{mathabx}
\usepackage{mathtools}

\usepackage{color}
\usepackage{framed}
\definecolor{shadecolor}{rgb}{0.9,0.9,0.9}

\usepackage{enumerate}
\usepackage{multicol}
\usepackage{graphicx}

\usepackage{hyperref}
\usepackage{skak}

\usepackage{geometry}
\geometry{tmargin=2cm,bmargin=4cm,lmargin=2cm,rmargin=2cm}

\newcommand{\N}{\mathbb{N}}
\newcommand{\Z}{\mathbb{Z}}
\newcommand{\Q}{\mathbb{Q}}
\newcommand{\R}{\mathbb{R}}
\newcommand{\leer}{\underline{\;\;\;\;}}

\newcommand{\datum}[1]{\hfill {#1}\\}

\theoremstyle{definition}

\newtheorem{defn}{Definition}
\newtheorem{satz}{Behauptung}
\newtheorem*{aufg}{Aufgabe}
\newtheorem*{frage}{Frage}
\newtheorem*{antw}{Antwort}
\newtheorem*{nota}{Notation}
\newtheorem*{bsp}{Beispiel}
\newtheorem*{exk}{Exkurs}
\newtheorem*{acht}{Achtung}

\newenvironment{exkurs}{\begin{shaded}\begin{exk}}{\end{exk}\end{shaded}}
\newenvironment{satzliste}{\begin{enumerate}[(i)]}{\end{enumerate}}
\newenvironment{beweisliste}{\begin{enumerate}[Zu (i):]}{\end{enumerate}}


\begin{document}

\title{Matheschülerzirkel Klasse 7/8}
\author{Tim Baumann}
\date{}
\maketitle

\begin{center}
  \includegraphics[scale=0.15]{gregor}
\end{center}

\begin{shaded}
  \begin{nota}
    Wir verwenden die folgenden Bezeichnungen:

    \begin{tabular}{ r l l }
      $\N$ & Menge der natürlichen Zahlen & $\{ 1, 2, 3, 4, ... \}$ \\
      $\Z$ & Menge der ganzen Zahlen & $\{ ..., -2, -1, 0, 1, 2, ... \}$ \\
      $\Q$ & Menge der rationalen Zahlen & $\tfrac{1}{3}, \tfrac{-17}{2}, 1 \in \Q$
    \end{tabular}
  \end{nota}
\end{shaded}


\datum{22. November 2013}

\section{Unmöglichkeitsbeweise über Invarianten}

% Wird demnächst nachgetragen!

Vor uns befindet sich ein leeres Schachbrett:

\begin{center}
  \newgame
  \showallbut{p,P,q,Q,b,B,k,K,r,R,n,N}
  \showboard
\end{center}

Vor uns liegt außerdem ein großer Haufen Dominosteine. Ein Dominostein ist genauso groß wie zwei Felder des Schachbretts. Wenn du magst, kannst du dir das Schachbrett, die Dominosteine und ein paar Tetris-Steine, die wir später noch brauchen werden, auf der Webseite \mbox{\url{http://timbaumann.info/mathezirkel-kurs/invarianten-spiele.html}} ausdrucken, ausschneiden und selbst mitknobeln.

\begin{frage}
  Ist es möglich, das Schachbrett mit Dominosteinen so zu belegen, dass jedes Feld bedeckts ist, keine zwei Dominosteine übereinander liegen und kein Stein über den Rand hinausragt? 
\end{frage}

\begin{antw}
  Ja. Lege in jede Zeile des Feldes 4 Dominosteine horizontal nebeneinander.
\end{antw}

Wir sägen nun aus dem Schachbrett die rechte untere Ecke, das Feld \emph{h1}, heraus.

\begin{frage}
  Ist es immer noch möglich, das Schachbrett wie beschrieben mit Dominosteinen zu belegen?
\end{frage}

\begin{antw}
  Nein. Das Schachbrett ohne rechte untere Ecke hat 63 Felder. Jeder Dominostein belegt genau zwei Felder. Wenn eine Überdeckung möglich wäre, so hätte das Schachbrett ohne rechte untere Ecke somit eine gerade Anzahl von Feldern. Also kann es keine Überdeckung geben.
\end{antw}

Während wir die erste Frage einfach positiv (bejahend) beantworten konnten, indem wir eine Überdeckung mit Dominosteinen angegeben haben, fällt uns die negative (verneinende) Antwort schwieriger: Wir mussten nämlich einen Grund finden, warum es eine solche Überdeckung nicht geben kann. Es reicht nicht aus, zu sagen, man habe keine Lösung gefunden. Es könnte ja immer noch sein, dass man sich nur ungeschickt angestellt hat und deshalb keine Lösung gefunden hat.

Wir sägen nun aus dem Schachbrett auch die linke obere Ecke, das Feld \emph{h8}, heraus.

\begin{frage}
  Wie immer: Gibt es nun eine Überdeckung des Schachbretts mit Dominosteinen?
\end{frage}

Das Schachbrett ohne die beiden Ecken hat wieder eine gerade Anzahl von Feldern, nämlich 62. Prinzipiell könnte also eine Überdeckung möglich sein. Aber wenn du versuchst, eine zu finden, wirst du feststellen dass, egal wie du dich anstellst, zwei Felder übrig bleiben. Du vermutest daher, dass es keine Lösung geben kann.

\begin{antw}
  Nein. Wenn man einen Dominostein auf das Brett legt, so bedeckt er, egal wie er liegt, ein weißes und ein schwarzes Feld. Die beiden Felder, die wir abgesegt haben, waren beides weiße Felder. Damit ist das um zwei Ecken verkleinerte Brett noch 30 weiße und 32 schwarze Felder. Jedes Mal, wenn wir einen Stein setzen, nimmt die Zahl der noch offenen weißen und die Zahl der noch offenen schwarzen Felder um je 1 ab. Nach drei gelegten Dominosteinen haben wir beispielsweise noch $30-3=27$ offene weiße und $32-3=29$ offene schwarze Felder. Zu jedem Zeitpunkt gibt es genau zwei schwarze unbedeckte Felder mehr als weiße unbedeckte Felder. Wenn alle weißen Felder bedeckt sind, gibt es also noch zwei schwarze offene Felder. Diese können aber nicht nebeneinander liegen, deshalb können sie nicht mit einem Domino überdeckt werden.
\end{antw}

Wären nicht zwei diagonal gegenüberliegende, sondern zwei Ecken, die an einer Seite liegen, herausgesägt worden, so wäre die Aufgabe lösbar gewesen. Bevor wir die Antwort etwas tiefer analysieren, wollen wir uns noch eine weitere Aufgabe anschauen:

\begin{aufg}
  Auf einer Insel leben 345 gelbe, 346 grüne und 347 blaue Chamäleons. Wann immer sich zwei Chamäleons gleicher Farbe begegnen, passiert nichts. Wenn sich aber zwei Chamäleons unterschiedlicher Farbe begegnen, so nehmen beide die dritte Farbe an. Beispielsweise hätten wir nach einem Treffen eines gelben und einer grünen Chamäleons nur noch 344 gelbe, 345 grüne, aber dafür 349 blaue Chameleons. Frage: Ist es möglich, dass zu einem Zeitpunkt genau gleich viele Chameleons jeder Farbe auf der Insel leben?
\end{aufg}

Grundsätzlich könnte diese Situation eintreten, da $345 + 346 + 347 = 1038$ durch $3$ teilbar ist. Wenn wir allerdings versuchen, eine Liste von Begegnungen zu erstellen, sodass nach diesen Begegnungen die Anzahl der Chamäleons jeder Farbe gleich ist, scheitern wir. Spoiler: Auch diese Aufgabe ist nicht lösbar.

Was haben diese Aufgaben gemeinsam? Zunächst haben wir eine Anfangssituation, beispielsweise das leere (verkleinerte) Schachbrett oder die Anzahlen der Fische jeder Farbe im Aquarium. Dann verändert sich die Ausgangslage durch Züge (das Legen eines Dominosteins) oder Ereignisse (Treffen von zwei Fischen). Die Frage in beiden Aufgaben ist, ob eine bestimmte Situation (alle Felder bedeckt bzw. gleich viele Fische von jeder Farbe) eintreten kann.

In beiden Aufgaben finden wir experimentell keine Lösung und suchen daher nach einem Grund, warum wir jedes solche Unterfangen von vornherein zum Scheitern verurteilt ist. In der Schachbrettaufgabe könnten wir dies begründen, indem wir alle Möglichkeiten ausprobieren. Davon gibt es allerdings ziemlich viele, sodass wir eine Antwort auf diesem Weg nur, wenn überhaupt, mit Hilfe eines Computers finden können. In der zweiten Aufgabe allerdings, gibt es (auf den ersten Blick) unendlich viele Möglichkeiten, wie sich Fische treffen können; wenn wir beispielsweise herausgefunden haben, dass wir mit 40 Treffen von Fischen die gewünschte Endsituation nicht erreichen können, so könnte uns das 41 Treffen zum Ziel führen.

Wir haben uns daher in der Aufgabe mit dem Schachbrett eines anderen Tricks bedient: Wir haben bemerkt, dass am Anfang das verkleinerte Brett $32 - 30 = 2$ schwarze Felder mehr besitzt als weiße Felder. Jedes Mal, wenn wir einen Dominostein gelegt haben, wurde ein weißes und ein schwarzes Feld verdeckt, also blieb die Differenz zwischen der Anzahlen der schwarzen offenen und weißen offenen Felder immer gleich. Wir haben also eine Zahl entdeckt, die wir für jedes unbedeckte, teilweise oder vollständig mit Steinen bedeckte Spielbrett ausrechnen können und die mit jedem weiteren platzierten Stein, egal wo er gelegt wird, gleich bleibt. Man sagt auch, dass diese Zahl unverändert (mit Fremdwort invariant) bleibt und nennt sie eine \emph{Invariante}. In der gewünschten Endposition, dass das ganze Brett belegt ist, wäre die Differenz zwischen den offenen schwarzen und offenen weißen Feldern gleich $0 - 0 = 0$. Diese Situation kann also beginnend bei unserer Anfangsposition nicht erreicht werden.

\begin{antw}
  Es ist nicht möglich, dass es irgendwann gleich viele Chamäleons von jeder Farbe gibt. Wir betrachten die Zahl $C$, die wir als Differenz zwischen der Anzahl der blauen und grünen Chamäleons festlegen. Zu Beginn ist $C = 347 - 346 = 1$. Wenn sich ein blaues und ein grünes Chamäleon treffen, so bleibt diese Zahl gleich. Wenn sich allerdings ein grünes und ein gelbes Chamäleon treffen, so nimmt die Zahl der grünen Chamäleons um eins ab, während die Zahl der blauen um zwei steigt. Insgesamt erhöht sich $C$ um drei. Wenn sich ein blaues und ein gelbes Chamäleon treffen, so sinkt $C$ um drei (mit ähnlicher Begründung). Die Zahl $C$ ist also nicht invariant. Aber wir stellen fest: Zu Beginn ist $C$ gleich 1, also nicht durch 3 teilbar. Wir wissen aber: Wenn eine ganze Zahl $k \in \Z$ durch 3 teilbar ist, so sind auch die Zahlen $k + 3$ und $k - 3$ durch 3 teilbar. Umgekehrt ist, wenn $k \in \Z$ nicht durch 3 teilbar ist, auch die Zahlen $k + 3$ und $k - 3$ nicht durch 3 teilbar. Also können wir folgern, dass nach jedem Treffen von zwei Chamäleons unsere Zahl immer noch \emph{nicht} durch 3 teilbar ist. Unsere Invariante ist hier also nicht die Zahl $C$ selbst, sondern die Tatsache, dass $C$ nicht durch 3 teilbar ist. In der gewünschten Endsituation wäre $C = 0$, da wir verlangen, dass die Zahl der blauen und grünen Chamäleons dann gleich ist. Aber 0 ist durch 3 teilbar! Folglich kann diese Situation nicht erreicht werden.
\end{antw}

Invarianten sind ein nützliches Mittel für Aufgaben obiger Art, bei denen man zeigen soll, dass eine bestimmte Situation nicht erreicht werden kann. Ein Nachteil dieser Technik ist es, dass Invarianten oft nicht offensichtlich sind, sondern es einiger Kreativität und Erfahrung bedarf, um sie zu finden. Bei der Schachbrettaufgabe könnte man feststellen, dass am Ende jedes Versuches zwei schwarze Felder übrig bleiben. Generell bietet sich an, wenn man so eine Aufgabe angeht, erst einmal rumzuprobieren und dabei Zahlen, die einem wichtig erscheinen, nach jedem Schritt aufzuschreiben und danach nach Mustern zu suchen.

Auch in der höheren Mathematik spielen Invarianten eine wichtige Rolle: Es gibt beispielsweise eine Teilgebiet, das sich mit Knoten befasst. Einen Knoten stellt man sich dabei als geschlossenes Seil im dreidimensionalen Raum vor. Wenn wir einen Knoten haben, so stellen sich Mathematiker die Frage, ob wir diesen Knoten nur durch Bewegen des Seiles (ohne Zerschneiden) diesen Knoten auflösen können, sodass er nur noch aus einer Seilschlinge besteht. Um zu beweisen, dass dies für manche Knoten nicht möglich ist, haben Mathematiker Invarianten gefunden, die etwas komplizierter als unsere bisher gesehene Invarianten sind und beim Umformen eines Knoten gleich bleiben.

% TODO: Tetris-Aufgabe


\datum{6. Dezember 2013}

\section{Rechnen mit Restklassen}

\subsection{Teilbarkeit}

\begin{defn}
  Eine Zahl $b \in \Z$ ist durch $a \in \Z$ \emph{teilbar}, wenn es eine Zahl $c \in Z$ gibt mit
  \[ a \cdot c = b. \]
\end{defn}

\begin{nota}
  Wir verwenden dann die Kurzschreibweise $a \divides b$, gesprochen "`$a$ teilt $b$"'. Wir sagen auch, dass $a$ ein \emph{Teiler} von $b$ ist oder dass $b$ ein Vielfaches von $a$ ist. Im Fall, dass die Zahl $a$ die Zahl $b$ \emph{nicht} teilt, d.\,h. keine Zahl $c \in \Z$ existiert mit $a \cdot c = b$, schreiben wir $a \not\divides b$.
\end{nota}

\begin{bsp}
  Folgende Aussagen stimmen:
  \begin{multicols}{5}
    \begin{itemize}
      \item $3 \divides 6$
      \item $5 \not\divides 13$
      \item $8 \divides -8$
      \item $3 \divides 12345$
      \item $2 \not\divides 1001$
    \end{itemize}
  \end{multicols}
\end{bsp}

\begin{exkurs}
  Der Ausdruck $a \divides b$ ist eine mathematische Aussage. Andere Beispiele für mathematische Aussagen sind:
  \begin{itemize}
    \begin{multicols}{4}
      \item $\sqrt{2}$ ist irrational
      \item $3 > 5$
      \item $n$ ist gerade
      \item $m$ ist eine Primzahl
    \end{multicols}
    \item Jeder Winkel lässt sich mit Zirkel und Lineal halbieren.
    \item Für $n \in \N$ mit $n \geq 2$ gibt keine Zahlen $a, b, c \in \N$, sodass $a^n + b^n = c^n$ stimmt.
    \item Es gibt unendlich viele Primzahlenzwillinge, das sind Primzahlen $p$ und $q$, mit $q = p + 2$.
  \end{itemize}

  Mathematische Aussagen sind entweder richtig oder falsch. Beispielsweise ist in den obigen Beispielen die erste wahr, die zweite falsch und über die nächsten beiden können wir nichts sagen, da sie Zahlen $n$ und $m$ beinhalten, die erst noch genauer definiert werden müssen. Die vorletzte Aussage trägt den Namen "`Fermats letzter Satz"' und ist richtig, doch hat es über 300 Jahre gedauert, bis sie bewiesen werden konnte. Von der letzten Aussage wird vermutet, dass sie stimmt, es existiert jedoch kein Beweis.
\end{exkurs}

\begin{frage}
  Gibt es eine Zahl $a \in \Z$, die Teiler von $0$ ist, d.\,h. $a \divides 0$?
\end{frage}

\begin{antw}
  Ja, wir können sogar jede beliebige Zahl $a \in Z$ nehmen: Setze $c \coloneqq 0$, dann ist $a \cdot c = a \cdot 0 = 0$ und somit ist die Definition erfüllt.
\end{antw}

\begin{frage}
  Gibt es andersherum eine Zahl $b \in \Z$, die durch $0$ teilbar ist, also $0 \divides b$?
\end{frage}

\begin{antw}
  Ja, aber nur die Zahl $0$ selber. Mit $a = 0$, ist für ein beliebiges $c$ nämlich $a \cdot c = 0 \cdot c = 0$, also muss $b = 0$ sein.
\end{antw}

\begin{exkurs}
  Sei $z \in \Z$ eine ganze Zahl. Wenn $z$ ungerade ist, so ist $z$ nicht durch $8$ teilbar. Um nicht immer "`wenn ..., dann ..."' schreiben zu müssen, verwenden Mathematiker folgende Notation:

  \[ z \text{ ist ungerade } \implies z \not\divides 8 \]

  Dabei stehen auf der linken und rechten Seite des $\Rightarrow$-Zeichens mathematische Aussagen $P$ und $Q$. Die Zeile $(P \Rightarrow Q)$ ist wiederum selbst eine mathematische Aussage, nämlich die Aussage, dass wenn $P$ stimmt, dann auch $Q$ stimmt. Dabei ist es wichtig, dass links $P$ steht und rechts $Q$, denn in unserem Beispiel stimmt die Aussage andersrum nicht: Wenn $z$ nicht durch $8$ teilbar ist, dann muss $z$ noch nicht unbedingt ungerade sein. Z.\,B. ist $z = 4$ nicht durch $8$ teilbar, aber gerade.

  Ein anderes Beispiel: Eine ganze Zahl $m$ ist ungerade, wenn die Zahl $(m+1)$ gerade ist. Andersrum ist $(m+1)$ gerade, wenn $m$ ungerade ist. Hier ist also die Umkehrung erfüllt, im Gegensatz zum vorherigen Beispiel. Also ist $m$ immer dann und nur dann ungerade, wenn $(m+1)$ ungerade ist. Mathematiker verwenden dafür eine besondere Notation:
  \[ n \text{ ist ungerade } \quad \iff \quad (n + 1) \text{ ist gerade } \]
  Auf beiden Seiten des $\Leftrightarrow$-Zeichens stehen dabei wieder mathematische Aussagen. Der Pfeil $\Leftrightarrow$ bedeutet, dass die linke Aussage nur dann stimmt, wenn die rechte Aussage stimmt.
\end{exkurs}

Wir wollen nun ein paar Tatsachen über die Teilbarkeit beweisen.

\begin{satz}
  Seien $n, p, q \in \Z$ ganze Zahlen. Dann gilt:
  \begin{satzliste}
    \item $n \divides p \text{ und } p \divides q \implies n \divides q$
    \item $n \divides p \implies n \divides (p \cdot q)$
    \item $n \divides p \text{ und } n \divides q \implies n \divides (p + q)$
  \end{satzliste}
\end{satz}

\begin{proof}
  \begin{beweisliste}
    \item Aus der Definition von Teilbarkeit wissen wir, dass es $c, d \in \Z$ gibt mit
    \[ n \cdot c = p \quad \text{und} \quad p \cdot d = q \]
    Um zu zeigen, dass $n \divides q$ gilt, müssen wir nach derselben Definition eine Zahl für die Leerstelle finden, sodass die Gleichung
    \[ n \cdot \leer = q \]
    erfüllt ist. Wir behaupten, dass die Zahl $(c \cdot d)$ dies leistet. Wir rechnen nämlich nach:
    \[ n \cdot (c \cdot d) = \underbrace{(n \cdot c)}_{= p} \cdot d = p \cdot d = q. \]
    Dabei haben wir im ersten Schritt das Assoziativgesetz gebraucht.
    \item Aus der Definition von Teilbarkeit erhalten wir ein $c \in \Z$ mit
    \[ n \cdot c = p. \]
    Wir müssen folgene Leerstelle sinnvoll ersetzen:
    \[ n \cdot \leer = p \cdot q. \]
    Wir nehmen dafür die Zahl $(c \cdot q)$ und rechnen
    \[ n \cdot (c \cdot q) = \underbrace{(n \cdot c)}_{= p} \cdot q = p \cdot q \]
    \item Aus der Definition erhalten wir $c, d \in \Z$ mit
    \[ n \cdot c = p \quad \text{und} \quad n \cdot d = q. \]
    Es soll folgende Gleichung gelten:
    \[ n \cdot \leer = p + q \]
    Wir setzen $(c+d)$ für $\leer$ und rechnen
    \[ n \cdot (c+d) = \underbrace{n \cdot c}_{= p} + \underbrace{n \cdot d}_{= q} = p + q. \]
    Im ersten Schritt haben wir dabei das Distributivgesetz angewendet.
  \end{beweisliste}
\end{proof}

\begin{acht}
  Folgendes Rechengesetz gilt \emph{nicht}:
  \[ p \divides n \text{ und } q \divides n \implies (p \cdot q) \divides n \]
  Ein Gegenbeispiel dafür ist $p = 4, q = 6, n = 12$.
\end{acht}

\begin{exkurs}
  In den natürlichen Zahlen $\N$, den ganzen Zahlen $\Z$ und den rationalen Zahlen $\Q$ gelten folgende Rechenregeln:
  \begin{itemize}
    \item \emph{Kommutativgesetz}: $a + b = b + a$ und $a \cdot b = b \cdot a$
    \item \emph{Assoziativgesetz}: $a + (b + c) = (a + b) + c$ und $a \cdot (b \cdot c) = (a \cdot b) \cdot c$
    \item \emph{Distributivgesetz}: $a \cdot (b + c) = a \cdot b + a \cdot c$
  \end{itemize}
  Es ist eine gute Übung, sich zu überlegen, welche Plus- und Mal-Zeichen wir durch Minus- und Divisions-Zeichen ersetzen dürfen, sodass die Regeln immer noch stimmen.
\end{exkurs}


\subsection{Restklassen}

Sei $n$ eine natürliche Zahl und $p, q$ natürliche Zahlen, die bei der Division durch $n$ den gleichen Rest $r$ haben, also
\begin{align*}
  p : n &= a \text{ Rest } r \\
  q : n &= b \text{ Rest } r
\end{align*}
für zwei Zahlen $a, b \in \Z$. Wir können dabei außerdem annehmen, dass $r$ eine Zahl zwischen $0$ bis $n - 1$ ist (warum?). Wenn wir obige Gleichungen umschreiben, erhalten wir
\begin{align*}
  p &= n \cdot a + r, \\
  q &= n \cdot b + r.
\end{align*}
Wir rechnen:
\[ p - q = (n \cdot a + r) - (n \cdot b + r) = n \cdot a + r - n \cdot b - r = n \cdot a + n \cdot b = n \cdot (a + b). \]
Wir haben also $n \cdot (a+b) = p - q$, folglich nach Definition von Teilbarkeit $n \divides (p - q)$. Dies ist unser erstes halbwegs interessantes Ergebnis: Zwei Zahlen $p$ und $q$, die bei Division durch $n$ den gleichen Rest haben, unterscheiden sich nur durch ein Vielfaches von $n$.

\begin{defn}
  Für zwei Zahlen $p$ und $q$, die sich nur durch ein Vielfaches von $n \in N$ unterscheiden (d.\,h. $n \divides (p-q)$) schreiben wir
  \[ p \equiv q \pmod{n}. \]
  Wir sprechen: "`$p$ ist gleich $q$ modulo $n$"'.
\end{defn}

\begin{nota}
  Falls $p \equiv q \pmod{n}$ nicht stimmt, schreiben wir $p \not\equiv q \pmod{n}$.
\end{nota}

\begin{bsp}
  Folgende Aussagen stimmen:
  \begin{itemize}
    \begin{multicols}{3}
      \item $1 \equiv 4 \pmod{3}$
      \item $0 \equiv 16 \pmod{8}$
      \item $-4 \equiv 3 \pmod{7}$
      \item $-1001 \equiv -1003 \pmod{2}$
      \item $4 \not\equiv 2 \pmod{4}$
      \item $0 \not\equiv -101 \pmod{3}$
    \end{multicols}
  \end{itemize}
\end{bsp}

\begin{acht}
  Wir dürfen den Teil in Klammern auf keinen Fall weglassen! Es gilt nämlich
  $4 \equiv 7 \pmod{3}$, aber nicht $4 \equiv 7 \pmod{6}$!
\end{acht}

\begin{satz}
  Sei $n \in \N$ und die Zahlen $a, a_1, a_2, b, b_1, b_2, c \in \Z$. Dann gilt:
  \begin{satzliste}
    \item $a \equiv a \pmod{n}$
    \item $a \equiv b \pmod{n} \text{ und } b \equiv c \pmod{n} \implies a \equiv c \pmod{n}$
    \item $a_1 \equiv a_2 \pmod{n} \text{ und } b_1 \equiv b_2 \pmod{n} \implies a_1 + b_1 \equiv a_2 + b_2 \pmod{n}$
    \item $a_1 \equiv a_2 \pmod{n} \text{ und } b_1 \equiv b_2 \pmod{n} \implies a_1 \cdot b_1 \equiv a_2 \cdot b_2 \pmod{n}$
  \end{satzliste}
\end{satz}

\begin{proof}
  \begin{beweisliste}
    \item Es ist $a - a = 0$ und somit gilt $n \divides (a - a))$, da $0$ von jeder beliebigen Zahl geteilt wird.
    \item Es gilt nach Vorraussetzung $n \divides (a-b)$ und $n \divides (b-c)$, also nach unserem Wissen über Teilbarkeit
    \[ n \divides \underbrace{\left((a-b) + (b-c)\right)}_{= (a - c)}. \]
    \item Nach Vorraussetzung gilt $n \divides (a_1 - a_2)$ und $n \divides (b_1 - b_2)$. Somit
    \[ n \divides \underbrace{\left((a_1 - a_2) - (b_1 - b_2)\right)}_{= (a_1 + b_1) - (a_2 + b_2)}. \]
    \item Nach Vorraussetzung gilt $n \divides (a_1 - a_2)$ und $n \divides (b_1 - b_2)$. Somit gilt auch
    \[ n \divides (a_1 - a_2) \cdot b_1 \quad \text{und} \quad n \divides a_2 \cdot (b_1 - b_2), \]
    also auch $n \divides ((a_1 - a_2) \cdot b_1 + a_2 \cdot (b_1 - b_2))$. Es gilt aber
    \[ (a_1 - a_2) \cdot b_1 + a_2 \cdot (b_1 - b_2) = a_1 \cdot b_1 - a_2 \cdot b_1 + a_2 \cdot b_1 - a_2 \cdot b_2 = a_1 \cdot b_1 - a_2 \cdot b_2, \]
    somit ist dies gleichbedeutend zu $n \divides (a_1 \cdot b_1 - a_2 \cdot b_2)$.
  \end{beweisliste}
\end{proof}

% Definition Restklasse
\iffalse
\begin{defn}
  Für $n \in \N$ und $r \in \{ 0, 1, ..., n-1 \}$
\end{defn}

\begin{bsp}
  Du kennst Restklassen aus dem Alltag: $4$ Uhr nachmittags und $16$ Uhr sind Bezeichnungen für diesselbe Uhrzeit, da
  \[ 16 \equiv 4 \pmod{12}. \]
\end{bsp}
\fi

\begin{satz}
  Für alle $n \in \N$ gilt
  \[ 10^n \equiv 1\underbrace{00\cdots0}_{n \text{ Nullen}} \equiv 1 \pmod{3} \]
\end{satz}

\begin{proof}
  Es gilt
  \[ \underbrace{99 \cdots 9}_{n \text{ Neuner}} = 3 \cdot \underbrace{33 \cdots 3}_{n \text{ Dreier}}, \]
  also $3 \divides 99 \cdots 9$ bzw. $99 \cdots 9 \equiv 0 \pmod{3}$. Wir rechnen:
  \[ 10^n \equiv \underbrace{99 \cdots 9}_{n \text{ Neuner}} + 1 \equiv 0 + 1 \equiv 1 \pmod{3}. \]
\end{proof}

Du kennst wahrscheinlich folgenden Test auf Teilbarkeit durch $3$: Er besagt, dass eine Zahl genau dann durch $3$ teilbar ist, wenn ihre Quersumme durch $3$ teilbar ist. Vielleicht hast du dich auch schon einmal gefragt, warum dieser Test funktioniert. Mit der Vorarbeit, die wir bisher geleistet haben, fällt ein Beweis nicht schwer:

\begin{satz}
  Sei $a \in \Z$ eine ganze Zahl, wobei die Ziffern von $a$ im Zehnersystem $a_n, ..., a_0$ seien, also $a = a_n \cdot 10^n + a_{n-1} \cdot 10^{n-1} + ... + a_1 \cdot 10 + a_0$. Die Quersumme von $a$ ist dann gebeben durch $\mathrm{QS}(a) = a_n + a_{n-1} + ... + a_1 + a_0$. Es gilt dann
  \[ a \equiv \mathrm{QS}(a) \pmod{3}. \]
\end{satz}

\begin{proof}
  Es gilt für alle $i$ zwischen $0$ und $n$
  \[ a_i \cdot 10^i \equiv a_i \cdot 1 \equiv a_i \pmod{3} \]
  durch Anwenden der letzten Behauptung und den Modulo-Rechenregeln. Also haben wir
  \[ a_n \cdot 10^n + ... + a_1 \cdot 10 + a_0 \equiv a_n + ... + a_1 + a_0 \equiv \mathrm{QS}(a) \pmod{3}. \qedhere \]
\end{proof}

Das ist nicht ganz die Behauptung des Quersummentests, allerdings ist $m \in \Z$ genau dann durch $3$ teilbar, wenn $m \equiv 0 \pmod{3}$. Falls aber $\mathrm{QS}(a)$ durch $3$ teilbar ist, so haben wir $\mathrm{QS}(a) \equiv 0$ und es folgt mit der ersten Modulo-Rechenregel schon
\[ a \equiv \mathrm{QS}(a) \equiv 0 \pmod{3}. \]


Es gibt auch einen ganz ähnlichen Test für Teilbarkeit durch $11$: Eine Zahl $a \in \Z$ ist genau dann durch $11$ teilbar, wenn die alternierende Quersumme durch $11$ teilbar ist. Sei $a = a_n \cdot 10^n + ... + a_1 \cdot 10 + a_0$, dann ist die alternierende Quersumme von $a$
\[ \mathrm{AQS}(a) = a_0 - a_1 + a_2 - a_3 + ... \pm a_n. \]
Wenn $n$ gerade ist, steht dabei ein Plus-Zeichen vor $a_n$, sonst ein Minuszeichen. Das Wort alternierend deutet an, dass wir abwechselnd die Ziffern, beginnend bei der letzten, dazuzählen und abziehen.

Zum Beispiel ist $\mathrm{AQS}(1234321) = 1 - 2 + 3 - 4 + 3 - 2 + 1 = 0$ und da $0$ durch $11$ teilbar ist, ist auch $1234321$ durch $11$ teilbar.

% TODO: Beweis des Quersummentests

\section{Nim-Spiele}

\datum{24. Januar 2014}

% TODO

\section{Vollständige Induktion}

\datum{7. und 21. Februar 2014}

Vollständige Induktion ist eines der grundlegenden mathematischen Beweisverfahren. Vollständige Induktion benutzt man immer dafür, um zu zeigen, dass eine bestimmte Aussage für alle natürlichen Zahlen gilt. Zum Beispiel:

\begin{aufg}
  Zeige, dass für alle natürlichen Zahlen $n \in \N$ die folgende Formel gilt:

  \[ \frac{1}{2^1} + \frac{1}{2^2} + \frac{1}{2^3} + ... + \frac{1}{2^{n-1}} + \frac{1}{2^n} = 1 - 2^n. \]
\end{aufg}

Wir können nun hergehen und diese Formel für spezielle Werte von $n$ nachrechnen, z.\,B. für $n=1$ oder $n=5$ oder (mithilfe eines Computers) für alle natürlichen Zahlen $n$ kleiner als eine Million.

Wir können uns auch intuitiv klarmachen, warum diese Formel gilt: Wir stellen uns einen Zahlenstrahl vor. Wir beginnen bei der Zahl $\frac{1}{2}$. Dann addieren wir $\frac{1}{4}$, also die Hälfte des Abstands von $\frac{1}{2}$ zu $1$. Wir befinden uns dann bei $\frac{3}{4}$. Dann addieren wir $\frac{1}{8}$, also die Hälfte des Abstandes von $\frac{3}{4}$ zu $1$. Wenn wir so weitermachen, halbiert sich der Abstand von unserer aktuellen Zahl zur Zahl $1$ in jedem Schritt. Und das ist ziemlich genau die Aussage der Aufgabe.

Nun ist aber das Nachrechnen der Formel für konkrete natürliche Zahlen kein Beweis der Aufgabe. Wir können schließlich nicht die Formel für alle natürlichen Zahlen durch einzelnes Nachrechnen prüfen, denn es gibt ja unendlich viele natürliche Zahlen. Die zweite Überlegung ist schon deutlich näher an einem mathematischen Beweis (der nachfolgende Beweis folgt in gewisser Weise sogar den gleichen Überlegungen). Wir wollen nun die Aufgabe mathematisch ganz korrekt durch vollständige Induktion beweisen. Um uns das Leben einfacher zu machen, führen wir davor aber noch etwas Notation ein:

\begin{shaded}
  \begin{nota}
    In obiger Aufgabe summieren wir
    \[ \frac{1}{2^1} + \frac{1}{2^2} + \frac{1}{2^3} + ... + \frac{1}{2^{n-1}} + \frac{1}{2^n} = 1 - 2^n, \]
    also alle Brüche $\frac{1}{2^k}$, wobei $k$ nacheinander die Werte $1$ bis $n$ annimmt. Für solche Summen verwendet man folgende abkürzende Schreibweise:
    \[ \sum_{k=1}^n \frac{1}{2^k} = \frac{1}{2^1} + \frac{1}{2^2} + \frac{1}{2^3} + ... + \frac{1}{2^{n-1}} + \frac{1}{2^n} = 1 - 2^n. \]

    Der griechische Buchstabe $\Sigma$ heißt Sigma, die Zeilen $k=1$ darunter und $n$ darüber bedeuten, dass $k=1$, dann $k=2$, usw. bis $k=n$ gilt. Die Variable $k$ wird auch Zählvariable genannt. Die Werte des Ausdrucks $\frac{1}{2^k}$ rechts neben $\Sigma$ werden für all diese Werte von $k$ aufaddiert.

    Mit dieser Notation wird die Formel einfacher und damit überschaubarer. Außerdem spart man sich viele Auslassungspunkte.

    Weitere Beispiele für die Verwendung dieser Summennotation sind:

    \begin{align*}
      \sum_{k=1}^5 k &= 1 + 2 + 3 + 4 + 5 = 15 \\
      \sum_{k=0}^n 2^i &= 2^0 + 2^1 + 2^2 + ... + 2^n \\
      \sum_{j=1}^k \frac{1}{j} &= \frac{1}{1} + \frac{1}{2} + \frac{1}{3} + ... + \frac{1}{k}
    \end{align*}

    Das letzte Beispiel zeigt, das wir für die Zählvariable auch andere Buchstaben außer $k$ benutzen können, und statt bis $n$ auch bis $k$ summieren können.
  \end{nota}
\end{shaded}

Wir können damit die obige Aufgabe folgendermaßen umschreiben:

\begin{aufg}
  Zeige, dass für alle natürlichen Zahlen $n \in \N$ die folgende Formel gilt:
  \[ \sum_{k=1}^n \frac{1}{2^k} = 1 - 2^n. \]
\end{aufg}

\begin{proof}
  Wir führen Induktion über $n$ durch:

  \emph{Induktionsanfang:} Die Formel stimmt für $n=1$, wie wir leicht nachrechnen:
  \[ \sum_{k=1}^n \frac{1}{2^k} = \sum_{k=1}^1 \frac{1}{2^k} = \frac{1}{2^1} = 1 - \frac{1}{2^1} = 1 - \frac{1}{2^n} \]

  \emph{Induktionsschritt:} Wir nehmen an, dass die Formel für eine bestimmte natürliche Zahl $n \in \N$ gilt und zeigen, dass die Formel dann auch für die nächstgrößere natürliche Zahl, also $n{+}1$, gilt. Dazu rechnen wir:
  \[ \sum_{k=1}^{n+1} \frac{1}{2^k} = \left( \sum_{k=1}^{n} \frac{1}{2^k} \right) + \frac{1}{2^{n+1}} = \left( 1 - \frac{1}{2^n} \right) + \frac{1}{2^{n+1}} = 1 - \frac{2}{2^{n+1}} + \frac{1}{2^{n+1}} = 1 - \frac{1}{2^{n+1}} \]
  Bei der ersten Gleichheit steht dabei auf beiden Seiten genau dasselbe, nur jeweils mit etwas anderer Notation. Links wurde die Summennotation für das Addieren von $n+1$ Zahlen verwendet, rechts nur für $n$ Zahlen. Dafür wurde rechts der letzte Summand extra hinzuaddiert. Die zweite Gleichheit folgt aus der Annahme (s.\,o.), dass die Formel für die Zahl $n$ stimmt. Wir konnten also die Formel direkt für den eingeklammerten Ausdruck anwenden. Die restlichen Gleichheiten sind Routine-Rechnungen.
\end{proof}

Warum nun ist dies ein korrekter Beweis für die Aufgabe? Nun, wir können dem Beweis direkt entnehmen, dass die Formel für die kleinste natürliche Zahl, also $n=1$ gilt. Das haben wir im Teil "`Induktionsanfang"' direkt nachgerechnet. Dann gilt aber auch die Formel für $n=2$, denn: Die Formel gilt für $n=1$ und das Argument im Teil "`Induktionsschritt"' sagt uns, dass deswegen auch die Formel für $1+1=2$ gilt. Dann gilt aber auch die Formel für $n=3$, denn: Die Formel gilt für $n=2$ und das Argument im Teil "`Induktionsschritt"' sagt uns, dass deswegen auch die Formel für $2+1=3$ gilt. Dann gilt aber auch die Formel für $n=4$, denn: Die Formel gilt für $n=3$ und das Argument im Teil "`Induktionsschritt"' sagt uns, dass deswegen auch die Formel für $3+1=4$ gilt. Und so weiter und so fort. Da wir so jede natürliche Zahl erreichen können, gilt die Formel für alle natürliche Zahlen.

Wir wollen nochmal allgemein zusammenfassen, wie ein Beweis durch vollständige Induktion aufgebaut ist. Wenn wir eine mathematische Aussage für alle natürlichen Zahlen $n \in \N$ beweisen wollen, dann reicht es aus, zu zeigen:

\begin{itemize}
  \item Die Aussage gilt für die kleinste natürliche Zahl $n$, also $n=1$ (Induktionsanfang)
  \item Wenn die Aussage für eine natürliche Zahl $n$ gilt, dann gilt sie auch für die nächstgrößere natürliche Zahl $n+1$ (Induktionsschritt).
\end{itemize}

\begin{exkurs}
  Vollständige Induktion wird oft schlicht auch nur Induktion genannt. Das Adjektiv "`vollständig"' dient zur Abgrenzung der mathematischen Induktion von der philosophischen Induktion, einem Prinzip des Schlussfolgerns, dass die Ableitung von abstrakten Gesetzmäßigkeiten von konkreten Beobachtungen beschreibt.
\end{exkurs}

Am besten versteht man Induktion und wann man sie wirkungsvoll einsetzt, indem man Aufgaben rechnet. Darum:

\begin{aufg}
  Zeige: Für alle natürlichen Zahlen $n \in \N$ gilt:
  \[ \sum\limits_{k=1}^n k = \frac{n \cdot (n+1)}{2}. \]
\end{aufg}

% Begriff einführen: Induktionshypothese/-annahme
% TODO: Gauss-Summe
% TODO: Ungleichungsaufgabe (verschobener Induktionsanfang)
% TODO: Turnieraufgabe
% TODO: Graphenaufgabe
% TODO: Kleinigkeiten: Induktionsschritt von $n-1$ auf $n$, starke Induktion

\end{document}