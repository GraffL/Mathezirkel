\documentclass{scrartcl}

\usepackage[utf8]{inputenc}
\usepackage[ngerman]{babel}
\usepackage[a5paper]{geometry}
\usepackage{parskip} % Keine Einrückung, aber Abstand zwischen Absätzen
\usepackage{graphicx} % Grafiken einbinden

\geometry{margin=1.5cm}

% Zu bestimmende Größen
\newcommand{\RE}{r_E} % Radius der Erde
\newcommand{\RM}{r_M} % Radius des Mondes
\newcommand{\RS}{r_S} % Radius der Sonne
\newcommand{\DM}{d_M} % Entfernung des Mondes
\newcommand{\DS}{d_S} % Entfernung der Sonne
\newcommand{\VM}{V_M} % Bahngeschwindigkeit des Mondes

\newcommand{\person}[1]{\textbf{#1}} % Personenname

% Einheiten
\newcommand{\km}{\,\mathrm{km}} % Kilometer
\newcommand{\hour}{\,\mathrm{h}} % Stunden
\newcommand{\minu}{\,\mathrm{min}} % Minuten
\newcommand{\seco}{\,\mathrm{s}} % Sekunden
\newcommand{\days}{\,\mathrm{d}} % Tage

\renewcommand\familydefault{\sfdefault} % Sans-Serif

\begin{document}
  %\title{Größen und Entfernungen im Sonnensystem}
  %\author{Tim Baumann}
  %\maketitle

  \begin{center}
    \Huge Größen und Entfernungen im Sonnensystem \\[1em]
  \end{center}

  Wir werden folgende Werte bestimmen:

  \begin{center}
    \begin{tabular}{r l}
      $\RE$ & Radius der Erde \\
      $\RM$ & Radius des Mondes \\
      $\RS$ & Radius der Sonne \\
      $\DM$ & Entfernung Erde-Mond \\
      $\DS$ & Entfernung Erde-Sonne \\
    \end{tabular}
  \end{center}

  Dabei werden wir bereits bekannte Werte verwenden, um die noch unbekannten näherungsweise zu berechnen. Wir betreiben ein Bootstrapping unseres Wissens von Entfernungen: Am Anfang wissen wir gar nichts. Beobachtungen am Sternenhimmel liefern uns dann Zusammenhänge zwischen Entfernungen und Größen in Form von mathematischen Formeln. Wenn wir eine Größe oder Entfernung bereits kennen, so können wir diese Gleichung nach der noch unbekannten Variable auflösen und durch Einsetzen der bekannten Werte ihren Wert bestimmen.

  \newpage

  \section{Der Radius der Erde}

  % XXX: Bild von Aristoteles
  \person{Aristoteles} (384-322 v. Chr.) vermutete aus folgendem Grund, dass die Erde eine Kugel ist:
  Er erkannte den Schattenwurf der Erde als Grund für Mondfinsternisse.
  Da währenddessen der Schatten der Erde auf dem Mond immer eine runde Form besitzt, glaubte er, die Erde müsse rund sein.
  Aristoteles wusste auch, dass es Sterne gibt, die man von Ägypten aus sehen kann, aber nicht von Griechenland aus.

  \person{Eratosthenes} (276-194 v. Chr.) hat folgende Beobachtung gemacht:

  Zur Sonnenwende, Mittagszeit, scheint die Sonne genau senkrecht in die Brunnen in Asène, sodass kein Schatten auf der Wasseroberfläche sichtbar ist.
  Am selben Tag, zu derselben Zeit schien die Sonne mit einem Winkel von etwa $7^\circ$ in einen Brunnen in Alexandria. Von Händlern kannte er die ungefähre Entfernung zwischen Alexandria und Asène.

  Asène (heute Assuad) liegt auf dem nördlichen Wendekreis.
  Beide Städte liegen im heutigen Ägypten und sind ca. 845km voneinander entfernt.
  % Quelle: http://www.distance.to/Alexandria-egypt/Assuan

  Wie konnte er aus diesen Daten und Beobachtungen den Erdradius $\RE$ bestimmen?

  % XXX: Skizze

  \vspace{1em}

  \begin{center}
    \includegraphics{egypt_sat.png}
  \end{center}

  \newpage

  \subsection{Lösung}

  Wir nehmen an, dass die Sonne so weit von der Erde entfernt ist, dass ihre Strahlen in erster Näherung parallel auf die Erde auftreffen.

  Dann sieht man anhand einer Skizze, dass der Winkel Alexandria-Erdmittelpunkt-Syène gleich $7^\circ$ beträgt.

  Also ist die Entfernung Alexandria-Syène die Länge eines $7^\circ$-Segments eines Kreises vom Radius $\RE$, als Formel

  \[ \frac{7^\circ}{360^\circ} \cdot 2 \pi \cdot \RE = 845 \km. \]

  Wir erhalten durch Umstellen

  \[ \frac{7^\circ}{360^\circ} \cdot 2 \pi \cdot \RE = 845 \km \cdot \frac{1}{2 \pi} \cdot \frac{360^\circ}{7^\circ} \approx 6916\km. \]

  In Wahrheit beträgt der Erdradius $6378 \km$ am Äquator und $6356 \km$ an den Polen. (Im Folgenden werden wir mit $\RE = 6370 \km$ weiterrechnen.)
  Der Rechenfehler beträgt also etwa $8 \%$.

  \newpage
  \section{Entfernung des Mondes}

  \person{Aristoteles} glaubte, dass der Mond ebenfalls kugelförmig ist.
  Das erschien ihm als die plausibelste Erklärung für die Mondphasen:

  \begin{center}
    \includegraphics[width=5cm]{lunar_phases.png}
  \end{center}
  % Quelle: https://simple.wikipedia.org/wiki/Phases_of_the_Moon#/media/File:Moon_Phase_Diagram_for_Simple_English_Wikipedia.GIF

  Eine Mondfinsternis tritt auf, wenn Sonne, Erde und Mond in dieser Reihenfolge in einer Achse stehen. Dann befindet sich der Mond im Schatten der Erde.

  \person{Aristarchos} beobachtete, dass eine Mondfinsternis etwa 3 Stunden dauert. (Genauer: Das ist die Zeit, die vergeht von dem Moment, wo der erste Schatten auf den Mond fällt, bis zu dem Zeitpunkt, an dem die ersten Sonnenstrahlen wieder auf den Mond treffen.)
  % XXX: Skizze

  Für eine Erdumrundung (die Zeit, die vergeht, bis Sonne, Erde und Mond zueinander wieder gleich positioniert sind) benötigt der Mond ca. 30 Tage.

  Wie konnte er aus diesen Informationen den Abstand $\DM$ des Mondes zur Erde bestimmen?

  \newpage
  \subsection{Lösung}

  Während der drei Stunden Finsternis legt der Mond auf seiner Bahn eine Entfernung von $2 \RE$ zurück. Seine Bahngeschwindigkeit beträgt also
  \[ \VM = \frac{2 \RE}{3 \hour} = \frac{2 \cdot 6370 \km}{3 \cdot 3600 \seco} \approx 1,18 \km / \seco. \]
  Seine Bahngeschwindigkeit kann man alternativ über seine Umlaufzeit und Entfernung berechnen:
  \[ \VM = \frac{2\pi \cdot \DM}{30 \days} \]
  Es folgt:
  \[ \DM = \frac{30 \days}{2 \pi} \cdot \VM = \frac{30 \cdot 24 \cdot 3600 \seco}{2 \pi} \cdot 1,18 \km / \seco \approx 487000 \km \]

  Tatsächlich beträgt die (mittlere) Distanz zum Mond $384400 \km$.

  \newpage
  \section{Größe des Mondes}

  Für einen festen Beobachter auf der Erde scheint es, als würde der Mond die Erde in ungefähr einem Tag umrunden (wegen der Rotation der Erde). Aristarchos beobachtete:

  Ein Monduntergang dauert ca. zwei Minuten.

  Wie konnte er daraus den Radius $\RM$ des Mondes bestimmen?

  \newpage
  \newpage{Lösung}

  Bei seinem täglichen scheinbaren Umlauf hat der Mond eine Bahngeschwindigkeit von
  \[ \tilde{\VM} = \frac{2\pi \cdot \DM}{1 \days} = \frac{2\pi \cdot 384400 \km}{24 \cdot 3600 \seco} \approx 28,0 \km / \seco. \]

  Wir können diese Bahngeschwindigkeit auch folgendermaßen berechnen: Während des Monduntergangs (also vom Zeitpunkt, an dem der Mond den Horizont berührt, bis zum Zeitpunkt, bis er vollends untergegangen ist) legt er (scheinbar) eine Strecke von $2 \cdot \RM$ zurück. Also ist
  \[ \tilde{\VM} = \frac{2 \cdot \RM}{2 \minu}. \]
  Es folgt
  \[ \RM = \tilde{\VM} \cdot \frac{2 \minu}{2} = 28,0 \km / \seco \cdot 60 \seco = 1680 \km. \]

  In Echt beträgt der mittlere Durchmesser $\RM = 1738 \km$.

  \newpage
  \section{Größe und Entfernung der Sonne}

  Aristarchos beobachtete, dass bei einer Sonnenfinsternis der Mond die Sonne von der Erde aus gesehen die Sonne ziemlich genau bedeckt.

  \begin{center}
    \includegraphics{solar_eclipse.jpg}
  \end{center}

  \vspace{4cm}

  Zwischenfrage: Was folgt daraus für das Verhältnis von Sonnengröße und Sonnenabstand? (Tipp: Strahlensatz)

  Wann genau ist Halbmond? Man könnte denken, dass Halbmond genau zwischen Voll- und Neumond auftritt. Das stimmt aber nicht, wie folgende Zeichnung verdeutlicht:

  \vspace{4cm}

  Beobachtungen ergeben, dass Halbmond eine halbe Stunde früher (bzw. später) eintritt, als man naiv erwarten würde.

  Kannst du daraus die Entfernung $\DS$ zur Sonne bestimmen?\\
  Was folgt für den Radius $\RS$ der Sonne? \\
  (Tipp: Berechne den eingezeichneten Winkel $\theta$ und verwende Trigonometrie)

  \newpage
  \subsection{Lösung}

  Zur Zwischenfrage: Aus dem Strahlensatz folgt
  \[ \frac{\RS}{\DS} = \frac{\RM}{\DM} = \frac{1738 \km}{384400 \km} = 0,00452. \]

  Es gilt
  \[ \theta = \frac{\pi}{2} - 2\pi \cdot 0,5 \frac{\hour}{30 \days} \]
  und
  \[ \cos \theta = \frac{\DM}{\DS}. \]
  Es folgt
  \[ \DS = \frac{\DM}{\cos \theta} = \frac{384400 \km}{\cos(\frac{\pi}{2} - 2\pi \cdot \frac{1}{2 \cdot 30 \cdot 24})} \approx 88.000.000 \km. \]

  Für den Radius der Sonne folgt
  \[ \RS = 0,00452 \cdot \DS = 400000 \km. \]

  Tatsächlich ist
  \[
    \DS = 149.600.000 \km, \quad
    \RS = 696.300 \km.
  \]

  Wir liegen also etwas daneben mit unserer Rechnung. (Vielleicht ist die Messung von einer Stunde auch nicht so ganz genau.) Aber zumindest stimmt die Größenordnung. Die Sonne ist also $696.300 \km / 6370 \km = 109$ Mal so groß wie die Erde.
  \person{Aristarchos} fand deshalb die Vorstellung, die Sonne würde um die Erde kreisen, lächerlich. Er vertrat deshalb das heliozentrische Weltbild -- 1700 Jahre vor Kopernikus!
\end{document}
