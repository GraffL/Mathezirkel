\documentclass{uebungszettel}
\graphicspath{ . }

\begin{document}

\maketitle{Klasse 7./8. -- Gruppe 3}{10.\ Januar 2014}

\begin{aufgabe}{Sauklaue}
  Benjamin hat mit Stift und Papier die Zahl
    \[ 19! = 19 \cdot 18 \cdots 3 \cdot 2 \cdot 1, \]
  die auch 19 Fakultät genannt wird, ausgerechnet. Leider hat er undeutlich geschrieben, sodass er drei Ziffern dieses Produktes nicht mehr lesen kann:

  \[ 19! = 121 \, 6 \underline{\,\,\,} \, \underline{\,\,\,} \, 100 \, 408 \, 832 \, 0 \underline{\,\,\,} 0 \]

  Kannst du ihm helfen, die Ziffern zu rekonstruieren?

  Hinweis: Taschenrechner\footnote{und andere elektronische Geräte, wie z.\,B. Handys, natürlich auch ;)} sind bei der Bearbeitung dieser Aufgabe nicht zugelassen!
\end{aufgabe}

% 4. Landeswettbewerb Mathematik Bayern, Runde 1, Aufgabe 2
\begin{aufgabe}{Produkte und Summen auf dem Schachbrett}
Die Felder eines Schachbretts sind in beliebiger Reihenfolge mit den Zahlen 1 bis 64 belegt. Man darf zwei Felder auswählen und bildet die Summe und das Produkt ihrer Zahlen. Danach wird die Zahl des einen Feldes durch die Einerziffer dieser Summe und die Zahl des anderen Feldes durch die Einerziffer dieses Produktes ersetzt. 

Kann man durch mehrfache Anwendung dieses Verfahren erreichen, dass auf allen Feldern die gleiche Zahl steht?

\emph{Hinweis: } Was passiert, wenn du zwei gerade Zahlen auswählst? Was bei zwei ungeraden Zahlen? Bei einer ungeraden und einer geraden Zahl?
\end{aufgabe}

% Bundeswettbewerb Mathematik 2002, 1. Runde
\begin{aufgabe}{Zwölfeck}
  Aus zwölf Strecken der Längen 1, 2, 3, ..., 12 wird irgendwie ein Zwölfeck gelegt. Zeige, dass es in diesem Zwölfeck stets drei aufeinander folgende Seiten gibt, deren Gesamtlänge größer als 20 ist.
\end{aufgabe}

% 8. Landeswettbewerb Mathematik Bayern, Runde 1, Aufgabe 1 (abgewandelt: 1001 für 2006)
\begin{aufgabe}{Papierschnitzel}
  Ein Stück Papier wird in 7 oder 10 Stücke zerschnitten. Nun wird eines der vorhandenen Stücke wieder wahlweise in 7 oder 10 Stücke zerschnitten. Dieser Vorgang wird mehrmals wiederholt.

  Kann man auf diese Weise 1001 Papierstücke erhalten?
\end{aufgabe}

\end{document}
